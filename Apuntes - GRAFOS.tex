\documentclass[12pt]{report}


\usepackage{fouriernc}%la fuente
%\usepackage[sc]{mathpazo} %antigua fuente

\usepackage[utf8]{inputenc}

\usepackage[a4paper,width=150mm,top=25mm,bottom=25mm]{geometry}



\usepackage{subfiles} %esto es para modularizar el overleaf
%para usar este paquete solamente hay que usar el comando
%\subfile{}

\usepackage{graphicx}
\usepackage{framed}
\usepackage[dvipsnames]{xcolor} %agrega mas colores para xcolor.

\usepackage{xparse}
\usepackage{xstring}

\usepackage{stmaryrd} %para poner el comando \mapsfrom "<---|"	

\usepackage{amssymb}

\usepackage{amsmath}

\usepackage{subfig}

\usepackage{mathrsfs} % para tener mas tipos de texto: \mathscr que es una letra mayuscula cursiva.

\usepackage{tikz-cd}

\usepackage{tkz-graph}%este paquete es para crear grafos con el ambiente \begin{tikzpicture}

\usepackage{caption}

\usepackage[shortlabels]{enumitem}

\usepackage{mathabx}
\let\widering\relax %esto es porque hay problemas con el comando \widering que se define en la fuenta fouriernc y en el paquete \usepackage{mathabx}

\usepackage[spanish,activeacute]{babel}

\usepackage{xparse}
\usepackage{xstring}

\usepackage{braket} %para definir \set , \Set y que los conjuntos se vean mas lindos

\usepackage{mathtools}

\usepackage[shortlabels]{enumitem}

\usepackage{hyperref}
\hypersetup{
    colorlinks,
    citecolor=red,
    filecolor=red,
    linkcolor=red,
    urlcolor=red
}

%%%%%%%%%%%%%%%%%%%%%%%%%%%%%%%%%%%%%%%%%%%%%
\usepackage{amsthm}

\theoremstyle{plain}
\newtheorem{theorem}{Teorema}[section]
\newtheorem{lemma}[theorem]{Lema}
\newtheorem{proposition}[theorem]{Proposición}
\newtheorem{proposition/definition}[theorem]{Proposición/Definición}
\newtheorem{corollary}[theorem]{Corolario}
\newtheorem{conjecture}[theorem]{Conjetura}
\newtheorem{afirmacion}[theorem]{Afirmación}
\newtheorem{recuerdo}[theorem]{Recuerdo}

\theoremstyle{definition}
\newtheorem{definition}[theorem]{Definición}
\newtheorem{hypothesis}[theorem]{Hipótesis}
\newtheorem{example}[theorem]{Ejemplo}
\newtheorem{obs}[theorem]{Observación}
\newtheorem{notation}[theorem]{Notación}
\newtheorem{remark}[theorem]{Comentario}


%por alguna razon el teorema $warning  est aen uso, asi que lo remuevo de maqnera trucha
\newtheorem{warn}[theorem]{\textbf{ADVERTENCIA}}
\renewenvironment{warning}{\begin{warn}}{\end{warn}}

%crear ejercicio
\newtheorem{exercise}[theorem]{Ejercicio}
%solución
\newenvironment{solution}{\begin{proof}[Solución]}{\end{proof}}





%como crear un nuevo ambiente de teorema o proposición que este sobreado con un recuadro de "color". primero hacemos

%\newenvironment{Theorem}{\colorlet{shadecolor}{color} \begin{shaded} \begin{theorem} }{ \end{theorem} \end{shaded} }

%Notar que primero hay que definir el color del sobreado con el comando
%"\colorlet{shadecolor}{color}" y luego hay que usar el environment "shaded". Adentro de este ponemos el environment que queremos, en nuestro caso queremos "pintar" el environment "\begin{theorem}".


%se puede cambiar la tonalidad de un color "yellow!80" es el color amarillo pero al 80%  y el 20% es mezclado con blanco, i.e. está aclarado. Pero "yellow!80!Black" es 80% amarillo y 20% negro, i.e. es obscurecido 20%.

\newenvironment{Definition}{\colorlet{shadecolor}{Apricot!12} \begin{shaded} \begin{definition} }{ \end{definition} \end{shaded} }

\newenvironment{Example}{\colorlet{shadecolor}{Goldenrod!16} \begin{shaded} \begin{example}}{ \end{example} \end{shaded}}

\newenvironment{Remark}{\colorlet{shadecolor}{Orchid!12} \begin{shaded} \begin{remark}}{ \end{remark} \end{shaded}}

\newenvironment{Warning}{\colorlet{shadecolor}{red!12} \begin{shaded} \begin{warning}}{ \end{warning} \end{shaded}}

\newenvironment{Conjecture}{\colorlet{shadecolor}{magenta!16} \begin{shaded} \begin{conjecture}}{ \end{conjecture} \end{shaded}}

\newenvironment{Theorem}{\colorlet{shadecolor}{OliveGreen!18} \begin{shaded} \begin{theorem}}{ \end{theorem} \end{shaded}}

\newenvironment{Lemma}{\colorlet{shadecolor}{LimeGreen!12} \begin{shaded} \begin{lemma}}{ \end{lemma} \end{shaded}}

\newenvironment{Proposition}{\colorlet{shadecolor}{Green!12} \begin{shaded} \begin{proposition}}{ \end{proposition}\end{shaded}}

\newenvironment{Corollary}{\colorlet{shadecolor}{TealBlue!16} \begin{shaded} \begin{corollary}}{ \end{corollary} \end{shaded}}

\newenvironment{Obs}{\colorlet{shadecolor}{Dandelion!22} \begin{shaded} \begin{obs}}{ \end{obs} \end{shaded}}

\newenvironment{Exercise}{\colorlet{shadecolor}{Lavender!12} \begin{shaded} \begin{exercise}}{ \end{exercise} \end{shaded}}

%%%%%COLORES%%%%%%%%%%%%
%Hay varios comandos del paquete Xcolor:
%\color{blue,green,red,yellow,orange,black,white,pink,purble,etc...} hace que todo el bloque de texto se transforme en este color, se puede encerrar entre {} el bloque de texto que uno quiere colorear
%\textcolor{color}{text} escribe el texto "text" en "color".
%\colorbox{color}{text} pinta un rectangulo de "color" detrás del "text".
%\shaded



%lista de colores base de xcolor, como son colores de la extension del paquetem, empiezan con la primera letra mayuscula: si usaramos solo el paquete {xcolor} entonces no sería necesario.

%red, Green (fluorecente), Blue (muy obscuro), Cyan, Magenta, Yellow, Black, Gray, lightgray, White, darkgray, lightgray, Brown, lime (este verde mas lindo manzana), olive (marron verdoso feo), Orange, pink, Purple, teal (verde marino), Violet

%marco los colores lindos: red, Cyan, Magenta, Yellow, Black, Gray, White,  lime, Orange, pink, teal, Violet

%Colores que incluye el paquete dvipsnames: Apricot (color beige), Brown, Goldenrod, JungleGreen, Salmon, Lavender, SpringGreen, Turquoise, Plum, Emerald, BurntOrange (naranja piola), ForestGreen (verde oscuro), BrickRed (rojo obscuro)


\newcommand{\red}[1]{\textcolor{BrickRed}{#1}}

			\newcommand{\comentario}[1]{\red{#1}}

\newcommand{\green}[1]{\textcolor{SpringGreen}{#1}}

\newcommand{\blue}[1]{\textcolor{Cyan}{#1}}

\newcommand{\darkblue}[1]{\textcolor{Cyan!70!Black}{#1}}

\newcommand{\yellow}[1]{\textcolor{yellow!80!Black}{#1}} %se puede cambiar la tonalidad de un color "yellow!80" es el color amarillo pero al 80%  y el 20% es mezclado con blanco, i.e. está aclarado. Pero "yellow!80!Black" es 80% amarillo y 20% negro, i.e. es obscurecido 20%.

\newcommand{\black}[1]{\textcolor{Black}{#1}}

\newcommand{\gray}[1]{\textcolor{Gray}{#1}}

\newcommand{\purple}[1]{\textcolor{Purple}{#1}}

\newcommand{\beige}[1]{\textcolor{Apricot}{#1}}

\newcommand{\darkgreen}[1]{\textcolor{ForestGreen}{#1}}

\newcommand{\pink}[1]{\textcolor{Lavender}{#1}}

\newcommand{\salmon}[1]{\textcolor{Salmon}{#1}}

\newcommand{\brown}[1]{\textcolor{RawSienna!50!Black}{#1}}

\newcommand{\white}[1]{\textcolor{White}{#1}}

\newcommand{\orange}[1]{\textcolor{BurntOrange}{#1}}













%%%%%%%%%%%%%%%%%%%%%%%%%%%%%%%%%%%%%%%%%%%%%




%grupos de matrices
%SL
\newcommand{\SL}[2]{\operatorname{SL}_{#1} ( #2)}
%GL
\newcommand{\GL}[2]{\operatorname{GL}_{#1} ( #2)}

%matriz identidad
\newcommand{\Id}{\operatorname{Id}}



%enteros Z
\newcommand{\integers}{\mathbb{Z}}
%racionales
\newcommand{\rationals}{\mathbb{Q}}
%naturales
\newcommand{\naturals}{\mathbb{N}}
%reales R
\newcommand{\reals}{\mathbb{R}}
%imaginarios
\newcommand{\complex}{\mathbb{C}}
%p-adicos
\newcommand{\padics}{\mathbb{Q}_p}
%enteros p-adicos
\newcommand{\padicintegers}{\mathbb{Z}_p}

%cuerpos finitos
%Fp
\newcommand{\Fp}{\mathbb{F}_p}
%Fq
\newcommand{\Fq}{\mathbb{F}_q}



%valor absoluto p-adico
\newcommand{\abs}[1]{\left \vert #1 \right \vert}
%valor absoluto p-adico
\newcommand{\Abs}[1]{\left \vert \left \vert #1 \right \vert \right \vert}
%valuacion p-adica
\newcommand{\val}[1]{\operatorname{val} (#1)}

%Hom
\newcommand{\Hom}{\operatorname{Hom}}

%imagen y núcleo
\newcommand{\Imagen}{\operatorname{Im}}
\newcommand{\Ker}{\operatorname{Ker}}

%coker
\newcommand{\Coker}{\operatorname{Coker}}

%limite inverso
\newcommand{\liminv}{\varprojlim}


%un poco de typeset para categorias
\newcommand{\catname}[1]{{\operatorfont\textbf{#1}}}


\renewcommand{\hat}[1]{\widehat{#1}}
\renewcommand{\bar}[1]{\overline{#1}}

%declaro un comando nuevo para escribir restricción de funciones
\newcommand\rest[2]{{% we make the whole thing an ordinary symbol
  \left.\kern-\nulldelimiterspace % automatically resize the bar with \right
  #1 % the function
  \vphantom{\big|} % pretend it's a little taller at normal size
  \right|_{#2} % this is the delimiter
  }}


%%%%   COMANDO ALGEBRA CONMUTATIVA   %%%%

%altura de un ideal:
\newcommand{\height}{\textsc{height}}

%Clausura topológica
\newcommand{\closure}[1]{\overline{#1}}

%longitud de un A-modulo. Notacion: \length_A M
\newcommand{\length}{\operatorname{length}}

%Anulador de un $A$-módulo.
\newcommand{\Ann}[1]{\operatorname{Ann} (#1)}

%Cuerpo de fracciones. Notacion $\FracField A$.
\newcommand{\FracField}[1]{\operatorname{Fr} (#1)}


%%%%%%%%%%%%%%%%%%%%%%%%%%%%%%%%%%%%






%%%%   COMANDO TEORÍA DE NÚMEROS  %%%%

%Discriminante
\newcommand{\discriminant}[1]{\mathfrak{d} (#1 )}

%%%%Ideales primos%%%
%escribe una letra en notación mathfrak, para denotar a un ideal o elemento primo.

\newcommand{\primo}[1]{\mathfrak{#1}}
\newcommand{\Primo}[1]{\mathfrak{\MakeUppercase{#1}}}

%anillo de enteros O_K
\renewcommand{\O}{\mathcal{O}}
%anillo de enteros con subindice de cuerpo (input, por ejemplo $K$).
\newcommand{\integralring}[1]{O_{#1}}

%caracteristica de un cuerpo Char k
\newcommand{\Char}[1]{\operatorname{Char} #1}

%traza. Notación \trace = Tr
\newcommand{\trace}{\operatorname{Tr}}

%Traza de extensiones. Notación \Tr L K \alpha = \operatorname{Tr}_{L/K} (\alpha)
\newcommand{\Tr}[1]{\operatorname{Tr}_{L/K} (#1)} %la extension es L/K por default
\newcommand{\tr}[3]{\operatorname{Tr}_{#1/#2} (#3)}

%Norma de extensiones. Notación \Norm L K \alpha = \operatorname{N}_{L/K} (\alpha)
\newcommand{\Norm}[1]{\operatorname{N}_{L/K} (#1)}%la extension es L/K por default
\newcommand{\norm}[3]{\operatorname{N}_{#1/#2} (#3)}


%discriminante de una forma bilineal simetrica. notacion \disc{B} = \operatorname{disc} ( B)
\newcommand{\disc}[1]{\operatorname{disc} (#1)}

%%%%%%%%%%%%%%%%%%%%%%%%%%%%%%%%%%%%




%%%%%%%%%%%%%COMANDO GRAFOS%%%%%%%%%%%%%

%\ceil funcion techo
\newcommand{\ceil}[1]{\left\lceil #1  \right\rceil}

%\floor funcion piso
\newcommand{\floor}[1]{\left\lfloor #1  \right\rfloor}

%diámetro de un grafo
\newcommand{\diam}[1]{\operatorname{diam} (#1)}

%radio de un grafo
\newcommand{\rad}[1]{\operatorname{rad}(#1)}

%Kappa: 
\newcommand{\Kappa}{\mathcal{K}}















%%%%%%%%%%%%%%%%%%%%%%%%%%%%%%%%%%%%



%%%%%%%%%%%%%%%%%%%%%%%%%%%%%%
%Creamos un ambiente para cada clase


\newcounter{numeroClase}%ponemos un contador que empieza en 0 y que cuenta el número de clase

%creamos una clase, i.e. ponemos un aseccion con el numero de clase y con un argumento obligatorio \Clase{argumento obligatorio} que es la fecha de la clase, por ejemplo 13/03/23.
\newenvironment{Clase}[1]{
	\stepcounter{numeroClase}
    \section{Clase \thenumeroClase: #1}
}{}

%%%%%%%%%%%%%%%%%%%%%%%%%%%%%%
%Cada dibujo se puede automatizar:
%1) necesitamos el archivo "Dibujo n.png" en la carpeta "Clase m", donde $n$ es el número del dibujo y $m$ es el número de la clase.

\newcounter{numeroDibujo}[numeroClase]

\renewcommand\thefigure{\thesection.\arabic{figure}}  

%el comando Dibujo tiene dos inputs \Dibujo{input 1}{input 2}, el primer input es el número de dibujo y el segundo es el caption de la figura.
\NewDocumentCommand{\Dibujo}{O{0.8} m}{
\stepcounter{numeroDibujo}
\begin{center}
\includegraphics[width=#1\columnwidth]{"./Editor de Grafos/Figuras/Clase \thenumeroClase /Dibujo \thenumeroDibujo .pdf"}
\captionof{figure}{#2}
\end{center}
}

%%%%%%%%%%%%%%%%%%%%%%%%%%%%%%%

\title{Apuntes - GRAFOS}
\author{Enzo Giannotta}






\begin{document}

\maketitle

%--------------------------------- ACA VA LA TABLA DE CONTENIDOS

\tableofcontents

%---------------------------------

\chapter{Parte I: Introducción a grafos}


Bibliografia: \cite{diestelGraphTheory}.



\section*{Evaluaciones:}

\textbf{Nota:} sumativa significa examen.

\begin{enumerate}
\item Sumativa 1 : Lunes 24 abril. 30 porciento.
\item Sumativa 2: Lunes 22 mayo: 25 porciento.
\item Sumativa 3: Jueves 6 julio: 25 porciento.
\item Talleres (usualmente los viernes): 20 porciento (hay que trabajar y entregar lo que se hizo en el taller, te pueden hacer resolver en clase al azar el ejercicio al siguiente taller).
\end{enumerate}


%%%%%%%%%%%
\Clase{16/03/23}   %%%%%%%%%%%%%%%%%%%%%%%%%%%%%%%%%%%%%%%%%%%%%%%%%%%%%%%%%%%%%
%%%%%%%%%%%

\begin{Definition}
Un \textbf{Grafo} es un par ordenado $G = (V,E)$, donde $V$ es un conjunto de \textbf{vértices} y $E$ es un conjunto de \textbf{Aristas}. Es decir, las aristas son pares $(v_1,v_2)$ con $v_1,v_2 \in V$. En un prinicipio si el grafo \textbf{no es dirigido}, no importa el orden de los vértices que aparece en un par $(v_1,v_2)$.
\end{Definition}

La manera de visualizar un grafo es dibujar cada vértice y unir dos pares de vértices $v_1,v_2 \in V$ por un segmento que representa la arista $(v_1,v_2)$.

\begin{example}
Sea $V = \{ 1,2,3,4,5\}$ y $E = \{ \{1,3\} , \{3,4\} , \{ 5,4\} , \{4,1 \} , \{ 1,2\}\}$


\Dibujo{Dibujo del grafo $V$.}
\end{example}


\begin{definition}
Para un grafo $G = (W,R)$, denotamos por $V(G)$ a $W$ (los vértices) y por $E(G) = R$ (las aristas).

El número de vértices de $G$ se denota como $\abs G$ o $\abs{V(G)}$, y se llama \textbf{orden} de $G$. El número de aristas lo denotamos como $\Abs G$ o simplemente $\abs{E(G)}$. Un grafo con orden $1$ o $0$ se llama \textbf{trivial}.
\end{definition}

\begin{definition}
Si $v \in V(G)$ y $e \in E(G)$, y además $v \in e$, decimos que $v$ es \textbf{incidente} en $e$ y viceversa, i.e. $e$ es incidente en $v$. Los dos vértices que inciden en una arista son sus \textbf{extremos}.

Dos vértices $x,y$ son \textbf{adyacentes} o \textbf{vecinos} si $(x,y) \in E$ (Otra notación: $xy \in E$ donde $xy$ es la arista).
\end{definition}

\begin{obs}
Si mi grafo tiene $n$ vértices, entonces tiene a lo sumo $\binom n 2$ aristas. Luego, la cantidad de grafos que se pueden construir es $2^{\binom n 2}$.
\end{obs}

\begin{definition}
Si en un grafo todo par de vértices es adyacente, decimos que el grafo es \textbf{completo}.
Notamos: $K_n$ para todo $n \geq 1$, al grafo completo con $n$ vértices.
\end{definition}

\Dibujo{Ejemplo de grafos completos de orden $1,2,3$ y $5$.}

\begin{definition}
Si un par de vértices no es adyacente, decimos que son \textbf{independientes} o \textbf{estables}.

Si $V' \subset V(G)$ es tal que cada par de vértices en $V'$ es independiente, entonces decimos que $V'$ es \textbf{independiente}.
\end{definition}

\begin{definition}
Sean $G1 = (V1,E1)$ y $G2 = (V2, E2)$ grafos, decimos que $\varphi : V1 \rightarrow V2$ es un \textbf{isomorfismo} si para todo par de vértices $x,y \in V1$ se tiene que $xy \in E1 \Leftrightarrow \varphi (x) \varphi (y) \in E2$. 

Usualmente no hacemos distinciones entre dos grafos isomorfos. De hecho en ese caso escribimos $G1 = G2$.
\end{definition}


\Dibujo{Ejemplo de isomorfismo de grafos.}
En este ejemplo un isomorfismo válido entre $G1$ y $G2$ es
\begin{align*}
\varphi : 1 &\mapsto A \\
			2 &\mapsto B \\
			3 &\mapsto C \\
			4 &\mapsto D
\end{align*}


\begin{definition}
Definimos:
\begin{itemize}
\item $G \cup G' := (V \cup V', E \cup E')$
\item $G \cap G' := (V \cap V' , E \cap E')$
\end{itemize}
\end{definition}

\Dibujo{Ejemplo de unión e intersección de grafos.}

\begin{definition}
Si $G \cap G' = \emptyset$, decimos que son \textbf{disjuntos}.

Si $V' \subset V$ y $E' \subset E$, decimos que $G'$ es \textbf{subgrafo} de $G$, y que $G$ es \textbf{supergrafo} de $G'$. Notamos $G' \subset G$. Si $G'$ es subgrafo de $G$ pero $G \neq G$, decimos que $G'$ es \textbf{subgrafo propio} de $G'$ y análogamente decimois que $G$ \textbf{supergrafo propio} de $G'$; notamos $G' \subsetneq G$.
\end{definition}

\begin{definition}
Sea $G' \subset G$, tal que $G'$ contiene todas las aristas $xy \in E$ tal que $x,y \in V'$. Decimos que $G'$ es un \textbf{subgrafo inducido} de $G$.

En este caso diremos que $V'$ \textbf{induce} $G'$ en $G$, y escribimos $G' = G[V']$ para un subconjunto $V' \subset V$.
\end{definition}

\Dibujo{Ejemplo de grafo inducido y no inducido.}







%%%%%%%%%%%
\Clase{18/03/23}   %%%%%%%%%%%%%%%%%%%%%%%%%%%%%%%%%%%%%%%%%%%%%%%%%%%%%%%%%%%%%
%%%%%%%%%%%

\begin{definition}
Si $U \subset V(G)$, escribimos $G \setminus U$ para denotar $G[V \setminus U]$. Es decir, $G \setminus U$ se obtiene de borrar los vértices de $U$ y sus aristas incidentes.
\end{definition}

\Dibujo{Ejemplo de $G \setminus U$.}

\begin{definition}
El \textbf{complemento} $\bar G$ de un grafo $G$, es el grafo con vértices $V(G)$ y que tiene una arista $xy$ si y solo si $xy \not \in E(G)$.
\end{definition}

\Dibujo{Ejemplo de complemento.}

Notar que en el ejemplo de arriba, $G$ y $\bar G$ son isomorfos. Esto no pasa necesariamente, por ejemplo el complemento de un grafo completo es el grafo sin aristas.


\subsection{El grado de un vértice}
\begin{definition}
Sea $G$ un grafo no vacío, y sea $v \in V(G)$. El conjunto de vecinos de $v$ lo denotamos como $N (v)$ o si el contexto no es claro $N_G (v)$. Llamamos a este conjunto el \textbf{vecindario} de $v$.

Más en general, si $U \subset V(G)$, no vacío. El \textbf{vecindario} de $U$ es el subconjunto de vértices de $V(G) \setminus U$ que contiene vecinos de algún elemento de $U$. Notamos $N(U)$ o $N_G (U)$.
\end{definition}

\Dibujo[0.6]{Ejemplo de vecindario de $U = \{ 3,4\}$. Tenemos que $N(U) = \{0,2\}$.}

\begin{Definition}
El \textbf{grado} de un vértice $v \in V(G)$ es el número de aristas que inciden en $v$ y lo denotamos como $d (v)$ o $d_G (v)$. Notar que 
$$
d(v) = \abs{N(v)},
$$
porque no permitimos multigrafos.

Si $v$ tiene grado $0$, decimos que es \textbf{aislado}.
\end{Definition}

\begin{Definition}
Definimos la cantidad de $G$:
$$
\delta ( G) := \min_{v \in V(G)} \{ d(v) \}.
$$
Es el \textbf{grado mínimo} de $G$.

Análogamente, tenemos la cantidad $G$:
$$
\Delta (G) := \max_{v \in V(G)} \{ d(v) \}.
$$
Es el \textbf{grado máximo} de $G$.

En el caso que todos los vértices tienen el mismo grado, i.e. $\delta (G) = \Delta (G)$, decimos que $G$ tiene grado $k$ y que $G$ es \textbf{$k$-regular} o simplemente \textbf{regular}.
\end{Definition}

\begin{definition}
Definimos la cantidad del grafo $G$:
$$
d(G) := \frac{1}{\abs {V(G)}} \sum_{v \in V(G)} d(v) = \frac{1}{\abs {G}} \sum_{v \in V(G)} d(v) .
$$
Es el \textbf{grado promedio} de $G$.
\end{definition}

\begin{Obs}
$$
\sum_{v \in V(G)} d(v) = 2 \abs{E(G)} = 2 \Abs G.
$$
Con lo cual,
$$
\boxed{d(G) = 2 \frac{E(G)}{V(G)}= 2 \frac{\Abs G}{\abs G}.}
$$
\end{Obs}

\begin{Proposition}
El número de vértices de grado impar en un grafo siempre par.
\end{Proposition}
\begin{proof}
Por la observación anterior, $\sum_{v \in V(G)} d(v) = 2 \abs{G} \equiv 0 \mod 2$ con lo cual,
$$
\# \set{ v \in V(G) | d(v) \equiv 1 \mod 2 } = \sum_{v | d(v) \equiv 1 \mod 2} d(v) \equiv 0 \mod 2 .
$$
\end{proof}

\begin{proposition}\label{prop:ultima proposicion de la clase anterior a caminos y ciclos}
Para todo grafo $G$ con al menos una arista, existe un subgrafo $H$ tal que
$$
\delta (H) > \frac{\abs{E(H)}}{\abs{V(H)}} \geq \frac{\abs{E(G)}}{\abs{V(G)}}.
$$
\end{proposition}
\begin{proof}
En efecto, la idea es la siguiente: construimos una secuencia de grafos $G = G_0 \supset G_1 \supset \ldots$ de subgrafos inducidos, tales que si $G_i$ tiene un vértice de grado $d(v_i) \leq \frac{\abs{E(G_i)}}{\abs{V(G_i)}}$, entonces tomamos $G_{i+1} := G_i \setminus v_i$; si no, la secuencia termina en $H := G_i$.
Por la elección de $v_i$ se sigue que $\frac{\abs{E(G_{i+1})}}{\abs{V(G_{i+1})}} \geq \frac{\abs{E(G_i)}}{\abs{V(G_i)}}$, pues esto sucede si y solo si
\begin{align*}
\frac{\abs{E(G_i)}-d(v_i)}{\abs{V(G_i)}-1} & \geq \frac{\abs{E(G_i)}}{\abs{V(G_i)}} \\
\Leftrightarrow (\abs{E(G_i)}-d(v_i))\abs{V(G_i)} &\geq \abs{E(G_i)}(\abs{V(G_i)}-1) \\
\Leftrightarrow -d(v_i) \abs{V(G_i)} &\geq -\abs{E(G_i)} \\
\Leftrightarrow \frac{\abs{E(G_i)}}{\abs{V(G_i)}} &\geq d (v_i).
\end{align*}
En particular, $\frac{\abs{E(H)}}{\abs{V(H)}} \geq \frac{\abs{E(G)}}{\abs{V(G)}}$.

Afirmamos que $H$ tiene al menos una arista, de lo contrario $\frac{\abs{E(H)}}{\abs{V(H)}} = 0 < \frac{\abs{E(G)}}{\abs{V(G)}}$, por tener $G$ al menos una arista. En particular, $H \neq \emptyset$. Como $H$ es el mínimo de esta construcción, se tiene que $\delta (H) > \frac{\abs{E(H)}}{\abs{V(H)}}$.
\end{proof}

\subsection{Caminos y Ciclos}

\begin{Definition}
Un \textbf{camino} es un grafo no vacío $P = (V,E)$ de la forma
$$
V = \{ x_0,x_1,\ldots,x_k\}, \ k \geq 0.
$$
Con
$$
E = \{ x_0x_1, x_1x_2,\ldots,x_{k-1} x_k\}.
$$
Donde todos los $x_i$ son distintos.

Decimos que $\Abs G$, i.e. el número de aristas, es su \textbf{longitud}.

Usualmente denotamos al camino $P$ como la secuencia de vértices
$$
P = x_0 x_1 \ldots x_k.
$$
En este caso diremos que $P$ es un \textbf{camino entre} $x_0$ y $x_k$.
\end{Definition}

Notar que $\abs G = k+1$, y que $\Abs G = k$.

\Dibujo[0.9]{Dibujo de un camino de $k+1$ vértices.}



\begin{definition}
Sea $C$ un grafo que se construye a partir de un camino $P = x_0x_1\cdots x_k$ con $k \geq 1$, en donde agregamos la arista $x_kx_0$. Este grafo se llama \textbf{ciclo}. Notamos a esta construcción $C:= P + x_k x_0$ o $x_0 x_1 \ldots x_k x_0$.

La \textbf{longitud} de un ciclo es su número de aristas, es decir $\Abs C$.
\end{definition}
Notar que $\abs C = k$.


\begin{definition}
Sea $G$ un grafo. Definimos la \textbf{cintura} de $G$ como la mínima longitud $g(G)$ de un ciclo en $G$.

Definimos la \textbf{circunferencia} como la máxima longitud de un ciclo en $G$.
\end{definition}

Si $G$ no tiene ciclos, definimos $g(G):= \infty$ y circunferencia $0$.

\Dibujo[0.6]{Ejemplo de ciclos en un grafo de cinutra igual a $4$.}


\begin{definition}
Una arista que une a dos vértices de un ciclo $C$, pero que no pertenece a $E(C)$, se la llama \textbf{cuerda}.
\end{definition}

\Dibujo{Ejemplo de dos cuerdas de un ciclo $C$.}

\begin{Proposition}\label{proposition:todo grafo tiene un camino de largo >= delta y ciclo de largo >= delta +1}
Todo grafo $G$ contiene un camino de largo $\geq \delta (G)$. Más aún, si $\delta (G) \geq 2$, entonces también contiene un ciclo de largo $\geq \delta (G) + 1$.
\end{Proposition}
\begin{proof}
Sea $P = x_0x_1\ldots x_k$ un camino de largo $k$ máximo en $G$. El caso $k = 1$ es inmediato, luego supongamos que $k \geq 1$.

\Dibujo[0.9]{Camino $P$ de longitud maximal $k$. Imaginemos que $x_0 = 0$ y que $x_k = 8$ (no pude cambiar las etiquetas en sage math cuando grafique el grafo).}

Notar que por maximalidad de $P$, todos los vecinos de $x_k$ están en $V(P)$, de lo contrario habria un camino más largo. Con lo cual 
$$
\abs{V(P)} = k \geq d(x_k) + 1 \geq \delta (G) + 1 .
$$
Con lo cual, $\Abs P \geq \delta (G)$.

Ahora, sea $i <k$ el menor índice tal que $x_i x_k \in E(G)$. Como $\delta (G) \geq 2$, se sigue que $i < k-1$, i.e. $x_i$ y $x_k$ no son adyacentes, luego tomamos el ciclo $C = x_i x_{i+1} \ldots x_k x_i$. Notar que entonces la longitud de $\Abs C \geq \delta (x_k) + 1 \geq \delta (G) + 1$.
\end{proof}



%%%%%%%%%%%
\Clase{20/03/23}   %%%%%%%%%%%%%%%%%%%%%%%%%%%%%%%%%%%%%%%%%%%%%%%%%%%%%%%%%%%%%
%%%%%%%%%%%


\begin{Definition}
La \textbf{distancia} entre dos vértices $x,y$ de un grafo $G$, es la longitud de un camino con longitud mínima entre $x,y$, la notamos
$$
d(x,y).
$$
Si no hay un camino entre $x$ e $y$, escribimos
$$
d(x,y) = \infty.
$$

El \textbf{diámetro} de $G$ es el máximo de las distancias entre todos los pares de vértices, lo notamos
$$
\diam G .
$$
\end{Definition}

Notar que $d(x,y) = 0$ si y solo si $ x = y$.

\begin{definition}
El \textbf{radio} de un grafo $G$, denotado $\rad G$, es la cantidad
$$
\rad G = \min_{x \in V(G)} \max_{y \in V(G)} d (x,y).
$$

Decimos que un vértice $v \in V(G)$ es \textbf{central}, si
$$
\max_{y \in v(G)} d(v,y) = \rad G. 
$$
Es decir, $v$ minimiza la función $x \mapsto \max_{y \in V(G)} d(x,y)$.

\Dibujo{Ejemplo de vértice central es $3$ y tiene radio $4$. El grafo $G$ tiene diámetro $8$.}
\end{definition}

\begin{Exercise}[\textbf{Entrega de taller}]
Probar que
$$
\rad G \leq \diam G \leq 2 \rad G .
$$
\end{Exercise}
\begin{solution}
Si el grafo $G$ no es conexo, luego el radio y el diámetro son infinito, luego vale la desigualdad. En efecto, por un lado si $x \in V(G)$ está fijo, como $G$ no es conexo existe $y \in V(G)$ tal que $d (x,y) = \infty$, con lo cual $\max_{y \in V(G)} d(x,y) = \infty$ para $x$ fijo, luego si tomamos mínimo sobre los $x$ se tiene que $\rad G = \infty$. Por otro lado, $\diam G = \infty$ porque es el máximo sobre todas las distancias entre dos vértices, y como mencionamos recién, al no ser conexo el grafo tiene que haber una distancia infinita entre algún par de vértices.

Ahora supongamos que $G$ es conexo, es decir para todo par de vértices $x,y$ existe un camino $P_{xy}$ que los conecta, sin pérdida de generalidad supongamos que es el más corto, i.e. $d(x,y)$ es la longitud de $P_{xy}$. Se deduce que $d(x,y) \leq \diam G$ por definición de diámetro. Tomando máximo sobre $y$ y luego mínimo sobre $x$ se sigue por definición que
$$
\rad G \leq \diam G.
$$
Esto prueba la primera desigualdad. Ahora veamos la segunda.

Sea $o$ un vértice que minimice la función $x \mapsto \max_{y \in V(G)} d(x,y)$, es decir, $o$ es central. Ahora tomemos dos vértices arbitrarios $x,y$. Como $o$  minimiza la función anterior, tenemos que $d(x,o) \leq \rad G = \max_{z \in V(G)} d(o,z)$, es decir existe un camino de longitud $\leq \rad G$ que une $x$ con $o$. Análogamente, existe un camino de longitud $\leq \rad G$ que une $o$ con $y$. Concatenando ambos caminos obtenemos un camino entre $x$ e $y$ de longitud $\leq 2 \rad G$. Tomando máximo sobre $x,y$ obtenemos la otra desigualdad:
$$
\diam G \leq 2 \rad G.
$$
\end{solution}

Si queremos relacionar el radio o diámetro con el grafo mínimo, promedio o máximo debemos tener otros parámetros como intermediario. Por ejemplo, los caminos tienen grado mínimo $1$ pero pueden tener radio y diámetro arbitrariamente grandes. O podemos tener radio y diámetro arbitrariamente grande y grado mínimo arbitrario. Antes de dar un ejemplo, necesitamos la siguiente definición:

\begin{definition}
Sea $G$ un grafo, definimos $G^k$ como la \textbf{potencia} de $G$. Es el grafo que contiene los mismos vértices y
las aristas son las originales pero agregando a cada vértice $x$ una arista incidente con cada vértice $y$ a
distancia $ d(x,y) \leq k$.
\end{definition}

\Dibujo[0.6]{El camino $P : 0,1,2,\ldots,7$ dibujado en \black{negro}, le agregamos aristas para dibujar $P^3$. Las aristas rojas conectan nodos a distancia $2$ y las \blue{azules} $3$.}

Por ejemplo, todo camino de longitud $2n$ tiene $2n+1$ vértices, radio $n$, diámetro $2n$. Luego $P^k$ tiene misma cantidad de vértices (pero no aristas), mismo radio y diámetro, pero grado $k$ para todo $1 \leq k \leq 2n$.

Vamos a relacionar el radio y grado máximo a través de el número de vértices. Un grafo puede tener muchos vértices, por ejemplo si tiene radio alto, o si tiene grado máximo alto, 

\begin{proposition}
Sea $d \geq 3$. Un grafo $G$ con radio a lo más $k$ y grado máximo a lo más $d$. Entonces tiene menos que $\frac{d}{d-2} (d-1)^k$ vértices.
\end{proposition}
\begin{proof}
Sea $z$ un vértice central de $G$ y $D_i$ el conjunto de los vértices a distancia $i$ de $z$.

\Dibujo{$D_0$ son los \green{verdes}, $D_1$ los \yellow{amarillos}, y así...}

Tenemos que $\abs{D_0} =1, \abs{D_1} \leq \Delta \leq d$. Notar que cada vértice de $D_1$ tiene como vecino en $D_2$ a lo sumo $d - 1$ vértices, pues ya es vecino de $z$. En general, tenemos que
$$
\abs{D_{i+1}} \leq \abs{D_i} (d-1) , \quad i \geq 1 .
$$
Con lo cual 
$$
\abs{D_{i+1}} \leq \abs{D_1} (d-1)^i = d (d-1)^i , \quad i \geq 1 .
$$
Entonces
\begin{align*}
\abs{V(G)} = \sum_{i=0}^k \abs{D_i} &\leq 1 + d \sum_{i=0}^{k-1} (d - 1)^i \\
&= 1 + d (d-1)^k - 1 / d-2 \\
&< \frac{d}{d-2} (d-1)^k .
\end{align*}
\end{proof}

\begin{obs}
\begin{enumerate}[1.]
\item Cuando el radio es $k=1$, por ejemplo en un grafo estrella, la cantidad de vértices es asintóticamente igual a $\frac{d}{d-2} (d-1)^k$ cuando $d \rightarrow \infty$.

\Dibujo[0.6]{Ilustración del grafo estrella.}


\item La cota no es para nada óptima para grafos de potencia $P^k$ de caminos. Por ejemplo, si $P$ tiene $2n+1$ vértices, $k = n$ y $d = k \geq 3$. Luego en el mejor de los casos con $d = 3$, tenemos que
$$
\abs{P^k} = 2n+1 \ll 3 \cdot 2^n .
$$
O sea que la diferencia es exponencial.

\end{enumerate}
\end{obs}

Similarmente, podemos acotar el orden de $G$ por abajo, si es que podemos controlar inferiormente $\delta$ y $g$. Definamos la cantidad para $d \in \reals$ y $g \in \naturals$:
\[
    n_0(d,g):= \begin{cases}
                1 + d \sum_{i = 0}^{r-1} (d-1)^i & \text{ si $g = 2 r +1$ es impar,}\\
                2 \sum_{i=0}^{r-1} (d-1)^i & \text{ si $g = 2 r$ es par.}
                \end{cases}
\]

\begin{theorem}[Versión débil]\footnote{La versión fuerte de este teorema, por Alon, Hoory and Linial, 2002, dice que
 si $d(G) \geq d \geq 2$ y $g(G) \geq g \in \naturals$, entonces $\abs G \geq n_0 (d,g)$.}
Sea $G$ un grafo con $\delta (G) \geq d \geq 2$ y $g(G) \geq g \in \naturals$. Entonces
\[
    \abs G \geq n_0 (d,g).
\]
En particular, $\abs G \geq n_0 ( d(G)/2, g)$.
\end{theorem}
\begin{proof}
Notar que la función es creciente en ambas variables para todo $d \geq 2$ y $g \in \naturals$. Con lo cual, basta
probar la afirmación para $d = \delta(G)$ y $g = g(G)$.

Sea $v$ un vértice de un ciclo $C$ de largo mínimo, i.e. $\geq g$. Consideremos como $D_i$ al conjunto de vértices a distancia $i$ de $v$ en $G$. Como antes, $\abs {D_0} = 1$, para cada vértice de $D_{i+1}$ tiene un vecino en $D_{i}$ si $i>0$; como cada vértice de $D_i$ tiene un vecino en $D_{i-1}$, se sigue que $\abs{D_{i+1}}  \geq (d-1) \abs{D_i}$ si $i < r$, ya que de lo contrario existiría un ciclo de longitud más chica que $r$.
Reiterando recursivamente esta igualdad, se sigue que $\abs{D_{i+1}} \geq (d-1)^i \abs{D_1}$ para todo $i < r$

Luego como
$$
G = \sqcup_{i} D_i \supset \sqcup_{0\leq i\leq r} D_i,
$$
se sigue que
\begin{enumerate}
\item[Caso $g = 2 r +1$]
$$
\abs G \geq \sum_{0 \leq i \leq r} \abs{D_i} =  1 + \sum_{i=0}^{r-1} \abs{D_{i+1}} \geq 1 + \sum_{i=0}^{r-1} (d-1)^i \abs{D_1} \geq 1 + \sum_{i=0}^{r-1} (d-1)^i d.$$

\item[Caso $g = 2 r$] Analogo.
\end{enumerate}

La última afirmación vale, pues sea $d = d(G)/2 = \epsilon (G)$, luego por la Proposición \ref{}, existe un subgrafo $H$ de $G$ tal que $\delta (G) > d$, y por lo tanto aplicando este teorema a $H$ se tiene que $\abs G \geq \abs H \geq n_0 (\delta(H),g(H))$, pero como $\delta (H) > d$ y $g(H) \geq g(G) = g$, y $n_0$ es creciente en ambas variables, se sigue que $\abs G \geq n_0 (d,g)$.

\end{proof}
\begin{corollary}
Si $\delta(G) \geq 3$, entonces $g(G) < 2 \log_2 \abs G$.
\end{corollary}
\begin{proof}
Tomamos $g = g(G)$. Si es par, entonces
$$
n_0(3,g) = 2 \frac{2^{g/2}-1}{2  - 1} = 2^{g/2} +(2^{g/2}-2) > 2^{g/2}.
$$
Si $g$ es impar, entonces
$$
n_0(3,g) = 1 + 3 \frac{2^{(g-1)/2}-1}{2-1} = \frac{3}{\sqrt 2} 2^{g/2} - 2 > 2^{g/2}.
$$
Luego por el teorema anterior el resultado se sigue luego de tomar logarítmo en base $2$.
\end{proof}

\begin{proposition}
Todo grafo $G$ que contiene al menos un ciclo, satisface
$$
g(G) \leq 2 \diam G +1.
$$
\end{proposition}
\begin{proof}
Supongamos que no. Es decir, si $C$ es el ciclo de $G$ con menor longitud, se tiene que $\Abs C = g(G) \geq 2 \diam
G+2$. Es decir, existen dos vértices de $C$, digamos $x,y$ tales que su distancia en $C$ es mayor o igual a $\diam G +1$. En $G$, estos vértices están a distancia menor que $\diam G+1$, sea $P$ el camino mas corto en $G$ que une a $x,y$ (i.e. tiene longitud $< \diam G +1$), luego $P$ no es subgrafo de $G$. Con lo cual, existe un subcamino de $P$ que es un $C$-camino; luego este camino unión el $x-y$ camino más corto de $C$ es un ciclo de longitud más chica que la de $C$, absurdo.
\end{proof}

\subsection{Conexidad}

\begin{Definition}
Un grafo es \textbf{conexo} si es no vacío y para todo par de vértices, existe un camino que los une a ambos.
\end{Definition}

\begin{Proposition}\label{prop: todo grafo conexo se puede enumerar de manera que G_i[v1,...,vi] es conexo para todo i}
Los vértices de un grafo conexo $G$ se pueden enumerar, digamos $v_1,v_2,\ldots,v_n$ tal que $G_i = G[v_1,v_2,\ldots,v_i]$ es conexo para todo $i = 1,\ldots ,n$.
\end{Proposition}

\begin{proof}
Probaremos la proposición por inducción en $n$. Sea $v$ arbitrario, y asumamos por inducción que $v_1,\ldots,v_i$ han sido escogidos para $i < \abs{V(G)}$ y que $G_i$ es conexo 

Escojo un vértice $v$ no enumerado aún. Como $G$ es conexo existe un camino $P$ entre $v_1$ y $v$.

\Dibujo[0.7]{Camino $P$ que une a $v$ y $v_1$. En color \blue{azul} representa los vértice de $G_i$ en $P$.}

Tomamos como $v_{i+1}$ al último vértice en $P$, contado desde $v$, que no está en $G_i$. Como $v_{i+1}$ tiene vecino en $G_i$ y $G_i$ es conexo se tiene que $G_{i+1}$ es conexo.

\end{proof}

\begin{example}
El siguiente dibujo es un mal ejemplo:

\Dibujo{Mal ejemplo, pues \gray{$0$ y $2$} no inducen un grafo conexo $G[0,2]$.}
\end{example}








%%%%%%%%%%%
\Clase{23/03/23}   %%%%%%%%%%%%%%%%%%%%%%%%%%%%%%%%%%%%%%%%%%%%%%%%%%%%%%%%%%%%%
%%%%%%%%%%%

\begin{definition}[Maximalidad]
Consideremos una propiedad $P$, para para algún grafo, conjunto de vértices, etc. Decimos que un conjunto de vértices $U$ es maximal para $P$, si $U$ cumple $P$, y $U \cup \{v\}$ con $v \not \in U$ no cumple $P$.
\end{definition}

\Dibujo{Ejemplo de camino maximal: $0,1,2,3,5,6,7,8$. Sin embargo, $5,7,8$ no lo es.}


\begin{definition}
Sea $G = (V,E)$ un grafo. Un subgrafo conexo maximal de $G$ es llamado una \textbf{componente} o \textbf{componente conexa} de $G$.
\end{definition}

\Dibujo{Ejemplo de componentes: tiene $3$.}


\begin{notation}
Sea $G= (V,E)$ un grafo. Vamos a notar a la cantidad de arisstas por vértice como la cantidad:
\[
    \epsilon ( G) = \frac{\abs E}{\abs V}.
\]
\end{notation}

\begin{theorem}
Sea $k \in \naturals$. Todo grafo $G$ con $d(G) \geq 4 k$ tiene un subgrafo $H$ que es $(k-1)$-conexo tal que $\epsilon (H) > \epsilon (G) -k$.
\end{theorem}
\begin{proof}
Notemos por $\gamma = \epsilon (G)$; como $\gamma = \frac 1 2 d(G)$, tenemos que $\gamma \geq 2 k$. Consideremos los subgrafos $G' \subset G$ tales que
\[
    \abs{G'} \geq 2k \quad \text{y} \quad \Abs{G'}> \gamma ( \abs{G'}-k);
\]
esta familia es no vacía, pues $G$ cumple esta condición; notemos por $H$ al subgrafo de menor orden que cumple esta condición.
En efecto, $G$ cumple ambas condiciones, pues por un lado $\Abs G > \gamma (\abs G - k) = \Abs G - \gamma k$, y por otro lado
$$
\Delta (G) \geq d(G) \geq 4k,
$$
con lo cual existe un vértice de $G$ de grado máximo, con al menos $4k$ vecinos, es decir $\abs G \geq 4k +1 \geq 2k$.

Notar que ningún grafo $G'$ que cumpla la condición tiene orden exactamente $2k$, ya que esto implicaría que $\Abs {G'} > \gamma k \geq 2k^2 > \binom{\abs {G'}}{2} = k(2k-1)$, lo cual contradice la cantidad máxima de aristas que puede tener un grafo de $\abs{G'}$ vértices.
Por minimalidad de $H$, se tiene que $\delta (H) > \gamma$: de lo contrario podríamos eliminar un vértice de grado a lo más $\gamma$ y obtener un grafo $G' \subsetneq H$ que satisface la condición. En particular, como existe un vértice de grado $> \gamma$, se sigue que $\abs H \geq \gamma$. Ddividiendo la desigualdad $\Abs H > \gamma \abs H - \gamma k$ de la c ondición por $\abs H$, tenemos que $\epsilon (H) > \gamma - k$, como queríamos.

Falta ver que $H$ es efectivamente $(k+1)$-conexo. De lo contrario, $H$ tendría una separación propia $\{U_1, U_2\}$ de orden a lo más $k$; escribamos $H[U_i] =: H_i$. Como todo vértice $v \in U_1 \setminus U_{2}$ tiene $d(v) \geq \delta (H) > \gamma$ vecínos de $H$ en $H_1$, tenemos que $\abs{H_1}\geq \gamma \geq 2k$. Similarmente, $\abs{H_2} \geq 2k$. Por la minimalidad de $H$, ninguno de los $H_1,H_2$ puede satisfaced la condición, con lo cual
\[
    \Abs {H_i} \leq \gamma (\abs{H_i}-k), \quad i = 1,2.
\]
Sin embargo, tenemos que
\begin{align*}
\Abs H &\leq \Abs {H_1} + \Abs{H_2} \\
    &\leq \gamma (\abs {H_1} + \abs{H_2} -2k) = \gamma (\abs{H}+ \abs{H_1 \cap H_2}-2k) \\
    &\leq \gamma (\abs H - k) \quad (\text{pues $\abs{H_1\cap H_2} \leq k$}),
\end{align*}
contradiciendo la condición.
\end{proof}





\subsection{Árboles y bosques}

\begin{Definition}
Un grafo acíclico, es decir, sin ciclos, es llamado un \textbf{bosque}.

A un bosque conexo lo llamamos \textbf{árbol}, es decir un grafo conexo y acíclico.

Los vértices de grado $1$ son sus \textbf{hojas}, los otros vértices son sus \textbf{vértices interiores}.
\end{Definition}
Notar que las componentes conexas de un bosque son árboles.

\begin{obs}
Los subgrafos conexos de un árbol son árboles.
\end{obs}

\Dibujo{Un \brown{árbol} con \darkgreen{hojas} y \green{vértices interiores}.}



\begin{exercise}
Todo árbol tiene al menos $1$ hoja. Más aún, si el árbol tiene más de un vértice entonces tiene al menos $2$ hojas. En particular los árboles tienen grado mínimo $\delta = 1$.
\end{exercise}
\begin{solution}
Sea $P$ un camino maximal en el árbol, y miro uno de los extremos (podría haber solo uno si el camino tiene un solo vértice). Ese extremo si tuviera grado $\geq 2$, el otro vecino debería caer en el camino por maximalidad, luego existe un ciclo, absurdo!. Más aún, si el árbol tiene más de un vértice entonces el camino maximal que tomamos tiene dos vértices, i.e. dos hojas.
\end{solution}

Esto implica lo mismo para caminos, pues los caminos son árboles.

\begin{Theorem}
Sea $T$ un grafo. Las siguientes definiciones son equivalentes:
\begin{enumerate}[(i)]
\item $T$ es árbol.
\item Cada par de vértices en $T$ están unidos por un único camino.
\item $T$ es conexo, pero $T \setminus e$ es disconexo para todo $e \in E(T)$. Es decir, es minimalmente conexo.
\item $T$ es conexo, pero $T \setminus v$ es disconexo para todo $v \in V(T)$ que no sea hoja.
\item $T$ es acíclico, pero $T \cup xy$ tiene un ciclo para cualquier par de vértices $x,y$ no adyacentes. Es decir, es maximalmente acíclico.
\end{enumerate}
\end{Theorem}
\begin{proof}
\begin{enumerate}
\item[]

\item[(i) $\Rightarrow$ (ii)] Si no, existe al menos un camino por ser conexo, luego si hay dos caminos distintos entonces podemos construir un ciclo.

\item[(ii) $\Rightarrow$ (iii)] Sea $e$ una arista entre $xy$, entonces $xy$ es un camino entre esos vértices, por hipótesis es el único, luego al quitarlo debe quedar disconexo, de lo contrario es que había otro camino.

\item[(iii) $\Rightarrow$ (iv)] Como $v$ no es una hoja, es vecino de al menos dos vértices distintos, dicamos $a,b$. Si todo camino entre $a$ y $b$ pasa por $v$, entonces quitar este vértice haría que $T$ fuera disconexo. Supongamos que existe un camino que une $a,b$ pero que no contiene a $v$. Luego si quitamos la arista $av$ o $vb$ el grafo sigue siendo conexo, pues tenemos un ciclo $a v b$, absurdo.

\item[(iv) $\Rightarrow$ (v)] $T$ es acíclico, pues de lo contrario podríamos podríamos quitar un vértice y que siga quedando conexo. En efeco, sea $C$ un ciclo en $T$, digamos con vértices $x_0,x_1,\ldots,x_n,x_0$ y $n \geq 2$. Si quitamos cualquier vértice $v$ de $C$, este queda conexo, pero veamos que $T$ también. De lo contrario, es que $v$ separa a $T$ en dos componentes conexas, es decir, todos los caminos entre $C$ y $T \setminus C$ pasan por $v$, con lo cual tomando otro vértice de $C$ que no sea $v$ y quitándolo, nos quedaría que $T$ menos ese punto ex conexo, absurdo.

Sean $x,y$ no adyacentes. Consideremos $P$ un camino entre $x,y$. Como no son adyacentes este camino necesariamente tiene al menos un vértice en intermedio, digamos $z$. Por hipótesis, si quitamos $z$ el grafo nos queda disconexo, y esto lo podemos hacer para cualquier $z \neq x,y$ en $P$. Con lo cual, $P \cup xy$ es un ciclo en $T \cup xy$.

\item[(v) $\Rightarrow$ (i)] Por hipótesis, $T$ es acíclico. Sean $x,y \in V(G)$. Por hipótesis tenemos que $T \cup xy$ tiene un cíclo $C_{xy}$, luego $C_{xy}$ debe contener la arista $xy$ pues $T$ es acíclico. Entonces $C_{xy} \setminus xy$ conecta a $xy$. Como $x,y$ eran arbitrarios, tenemos que $T$ es conexo.
\end{enumerate}
\end{proof}

\begin{definition}
Sea $G$ un grafo. Un \textbf{árbol generador} de $G$ es un subgrafo de $G$ que es árbol y que contiene a todos los vértices de $G$.
\end{definition}

\Dibujo[0.35]{Ejemplo: árbol generador, con aristas en \red{rojo}.}

\begin{obs}
Del dibujo anterior podemos ver que el árbol generador no necesariamente es único, y de hecho, puede haber otro que no sea isomorfo. Por ejemplo cambiemos los vértices del árbol en \red{rojo}, de manera que ya no tenga vértices de grado $3$:
\Dibujo[0.20]{Otro árbol generador.}
\end{obs}

\begin{proposition}
Todo grafo conexo tiene un árbol generador.
\end{proposition}
\begin{proof}
Tomemos un subgrafo minimalmente conexo $H$, que contenga a todo $V(G)$. Por (iii) del teorema anterior tenemos que $H$ es un árbol. También se puede usar el ítem (v), aplicado a un subgrafo acíclico maximal.
\end{proof}

\begin{obs}
Esta demostración nos da un álgoritmo para construir el árbol generador de un grafo: quitamos aristas hasta que nos quede minimalmente conexo.
\end{obs}

\begin{definition}
Si $T$ es un árbol generador de un grafo $G$, las aristas en $E(G) \setminus E(T)$ son las \textbf{cuerdas} de $T$ en $G$.
\end{definition}

\begin{Proposition}
Los vértices de un árbol $T$ pueden ser enumerados, digamos $v_1,v_2,\ldots,v_n$ de manera que para todo $i \geq 2$, $v_i$ es hoja en $T[v_1,\ldots,v_i]$ (que es árbol también).
\end{Proposition}
\begin{proof}
Por la Proposición \ref{prop: todo grafo conexo se puede enumerar de manera que G_i[v1,...,vi] es conexo para todo i} existe una enumeración $v_1,\ldots,v_n$ tal que $T[v_1,\ldots,v_i]$ es conexo para todo $i \geq 1$. Inspeccionando la demostración, se puede ver que esta construcción sirve. En efecto, sabemos que $v_{i+1}$ tiene un vecino en $G[v_1,\ldots,v_i]$, llamemosló $x$, si tuviera otro llamado $y$, entonces $G[v_1,\ldots,v_i]$ contiene un camino $P_{xy}$ entre ellos, que junto con $x v_{i+1} y$ nos forma un cíclo, absurdo.

\Dibujo{Ver dibujo.}
\end{proof}








%%%%%%%%%%%
\Clase{27/03/23}   %%%%%%%%%%%%%%%%%%%%%%%%%%%%%%%%%%%%%%%%%%%%%%%%%%%%%%%%%%%%%
%%%%%%%%%%%


\begin{corollary}
Un grafo de $n$ vértices conexo es un árbol si y solo si tiene $n-1$ aristas.
\end{corollary}
\begin{proof}
\begin{enumerate}
\item[$\Rightarrow$)] Supongamos que $T$ es árbol; en particular es conexo. Por la proposición anterior, existe una enumeración $v_1,v_2,\ldots,v_n$ tal que para todo $i \geq 1$, el grafo $T[v_1,\ldots, v_i]$ tiene $i-1$ aristas por inducción.

\item[$\Leftarrow$)] Como $T$ es conexo tiene árbol generador $T'$, y por la implicación anterior tiene $n-1$ aristas, entonces $T = T'$, i.e. $T$ es árbol.
\end{enumerate}
\end{proof}

\begin{corollary}
Todo grafo conexo de $n$ vértices tiene al menos $n-1$ aristas.
\end{corollary}
\begin{proof}
Tiene un árbol generador, que debe tener $n-1$ aristas.
\end{proof}

\begin{corollary}
Si $T$ es un arbol y $G$ es un grafo con $\delta (G) \geq \abs T -1$, entonces $T \subset G$, es decir $G$ tiene un subgrafo isomorfo a $T$.
\end{corollary}
\begin{proof}
Sea $v_1,\ldots,v_n$ con $n = \abs T$, una numeración de los vértices de $T$ tal que $v_i, i \geq 2$ es una hoja de $T[v_1,\ldots,v_{i-1}]$. Haremos inducción en estos vértices. El caso base es trivial. Veamos el paso inductivo: supongamos que $G$ tiene a $T[v_1,\ldots,v_{i-1}]$ como subgrafo con $i\leq n$, entonces como $\delta (G) \geq n-1$, tenemos que $v_j, 1 \leq j \leq i-1$ (el único vecino de $v_i$) tiene al menos $\delta(G) \geq n-1$ vecinos en $G$, es decir, tiene un vecino que no está en $T[v_1,\ldots, v_{i-1}]$, luego agregamos a ese vértice como $v_i$ junto con la arista que le corresponde a $v_j$.
\end{proof}

\bigskip

De véz en cuando, es útil fijar un vértice $r$ de un árbol $T$, que llamaremos \textbf{raíz}. Un árbol con una raíz fija, se denomina \textbf{árbol enraigado} (o en inglés, \textbf{rooted tree}).
Recordemos que para todo $y \in V(T)$, existe un único camino entre $r$ e $y$ que denotaremos $rTy$; esto induce un orden parcial $x \leq y$ si y solo si $x \in r T y$. Este orden en $V(T)$ se lo llama el \textbf{orden del árbol} asosciado a $T$ y $r$. Definimos los conjuntos:
\[
    \ceil y := \Set{x | x \leq y} \quad \text{y} \quad \floor x := \Set{ y | y \geq x}
\]
la \textbf{clausura inferior} de $y$, y la \textbf{clausura superior} de $x$, respectivamente. En general, definimos $\ceil X := \bigcup_{x \in X} \ceil x $ y análogamente $\floor X$, para un conjunto $X \subset V(T)$. Un conjunto $X$ que coincide con $\ceil X$, se dice \textbf{cerrado inferiormente}, análogamente en el otro caso tenemos que es \textbf{cerrado superiormente}.

Notar que $r$ es el \textit{mínimo} en este orden, y todas las hojas de $T$ son \textit{maximales}. Los extremos de
una arista son siempre comparables entre sí, y los elementos de una clausura inferior forman una \textit{cadena} (i.e
. son comparables entre sí), sin embargo los elementos de una clausura superior no tienen por qué formar una cadena.
Decimos
que los
vértice a
distancia $k$ de $r$
tienen \textbf{altura} $
k$ y el
conjunto de estos vértices forma el $k$-ésimo \textbf{nivel} de $T$.

Un árbol enraigado $T$ contenido en un grafo $G$ se dice \textbf{normal} en $G$, si los extremos de todo $T$-camino en $G$ son comparables en el orden de $T$. Si $T$ genera $G$, esto equivale a pedir que dos vértices de $T$ sean comparables siempre que sean adyacentes en $G$; ver la siguiente figura:

\Dibujo{Un \brown{árbol} generador normal con raíz \darkgreen{r} de un grafo $G$.}

Un arbol normal puede ser una herramienta realmente útil para examinar la estructura de su grafo subyacente, ya que este grafo refleja las propiedades de separación de $T$:

\begin{lemma}
Sea $T$ un árbol normal en $G$. Tenemos que:
\begin{enumerate}[(i)]
\item Dados $x,y \in V(T)$, están separados en $G$ por el conjunto $\ceil x \cap \ceil y$.
\item Supongamos que $T$ genera $G$. Si $S \subset V(T) = V(G)$ y $S$ es inferiormente cerrado, luego las componentes conexas de $G \setminus S$ están generadas por los conjuntos $\floor x$ con $x$ minimal en $T \setminus S$.
\end{enumerate}
\end{lemma}
\begin{proof}
\begin{enumerate}[(i)]
\item Sea $P$ cualquier camino entre $x,y$ en $G$; veamos que $P$ interseca $\ceil x \cap \ceil y$. Sea $t_1,\ldots,t_n$ una sucesión de vértices en $P \cap T$, minimal con la propiedad que $t_1 = x$ y $t_n = y$ y $t_i,t_{i+1}$ son comparables en el orden del árbol $T$ para todo $i$. Dicha sucesión existe, pues el conjunto de todos los vértices en $P \cap T$, con el orden natural inducido por $P$, tiene esta propiedad, ya que como $T$ es normal todo segmento $t_i P t_{i+1}$ es una arista de $T$ o un $T$-camino. En nuestra secuencia minimal no podemos tener $t_{i-1}< t_i > t_{i+1}$ para ningún $i$, pues $t_{i-1},t_{i+1}$ son comparables y podríamos eliminar a $t_i$ de nuestra secuencia, obteniendo así una secuencia más chica. Entonces nuestra secuencia tiene la forma
\[
    x = t_1 > \ldots > t_k < \ldots < t_n = y
\]
(se podrían retirar los elementos en el ''medio''). Así, $t_k \in \ceil x \cap \ceil y \cap V(P)$.
\item Consideremos una componente $C$ de $G \setminus S$, y tomemos $x$ un elemento minimal ($T$ genera $G$) de $V(G)$. Afirmamos que $x$ es único, en efecto, si $x'$ fuera otro, ambos serian no comparables entre sí, pero por el ítem (i), cualquier camino entre $x,x'$ contiene un vértice más chico que ambos, contradiciendo minimalidad en $C$. Por lo tanto todo vértice de $C$ yace arriba de $x$: nuevamente por el ítem (i) hay un vértice debajo de ambos que por minimalidad es $x$. Recíproccamente, todo vértice $y \in \floor x$ está en $C$, pues como $S$ es cerrado inferiormente, el camino creciente $x T y$ yace en $T\setminus S$. Consecuentemente, $V(C)= \floor x$.

Ahora veamos que $x$ es minimal no solo en $V(C)$, sino también en $T \setminus S$. Los vérticces por debajo de $x$ forman una cadena $\ceil t$ en $T$. Como $t$ es vecino de $x$ en $T$, la maximalidad de $C$ como componente conexa de $G \setminus S$ implica que $t \in S$, y por lo tanto $\ceil t \subset S$ porque $S$ es cerrado inferiormente. Esto completa la demostración de que toda componente de $G\setminus S$ está generada por un conjunto $\floor x$ con $x$ minimal en $T \setminus S$.

Recíprocamente, si $x$ es un elemento minimal en $T \setminus S$, claramente también es minimal en la componente $C$ de $G \setminus S$ que lo contiene. Eso significa que $V(C) = \floor x$.
\end{enumerate}
\end{proof}


\begin{proposition}
Todo grafo conexo contiene un árbol generado normal, con el orden inducido por cualquier vértice como raíz.
\end{proposition}
\begin{proof}
Sea $G$ un grafo conexo y $r \in G$ un vértice fijo. Sea $T$ un árbol normal maximal con raíz $r$, veamos que $V(T) = V(G)$, i.e. genera $G$..

Supongamos por el absurdo que no, y sea $C$ una componente conexa de $G \setminus T$. Como $T$ es normal, la vecindad $N_G (C)$ (que está contenida en $T$) es una cadena en $T$, obviamente porque $C$ permite construir $T$-caminos entre cualquier par de vértices de $N_G(C)$. Sea $x$ su máximo elemento (recordemos que estamos en una cadena), y sea $y \in C$ adyacente a $x$. Sea $T'$ el arbol obtenido de $T$ agregando la arista $yx$; el orden de árbol de $T'$ extiende al de $T$. Veamos que $T'$ también es normal en $G$, contradiciendo maximalida.

Sea $P$ un $T'$-camino en $G$. Si sus extremos están en $T$, luego son comparables con el orden de $T$, y por lo tanto por el de $T'$ también, ya que $P$ es un $T$-camino también porque $T \subset T'$ y $T$ es normal. Si alguno de los extremos de $P$ fuera $y$, tenemos que $P \subset C$ salvo por su otro extremo $z$, que yace en $N_G(C)$. Como $x$ era máximo, tenemos que $z \leq x$. Luego $z,y$ serán comparables si vemos que $x < y$, es decir que $x \in r T' y$. Lo cual es claro ya que $y$ es una hoja de $T'$ con vecinon $x$.
\end{proof}




\subsection{Grafos bipartitos}

\begin{Definition}
Sea $r \geq 2$ entero. Decimos que un grafo $G = (V,E)$ es \textbf{$r$-partito} si podemos particionar a $V$ en $r$ partes tal que cada arista tiene sus extremos en partes distintas. Es decir, cada parte es un conjunto independiente.

$2$-partito es \textbf{bipartito}, $3$-partito es \textbf{tripartito}, etc.
\end{Definition}

\Dibujo[0.20]{Ejemplo de grafo $3$-partito, $4$-partito y $5$-partito, pero no $2$-partito porque siempre existirian dos vértices del triangulo \yellow{amarillo} en la misma partición, pero eso es imposible porque son adyacentes. Se ilustran dos triparticiones distintas: $A,B,C$ y por otro lado \red{rojo}, \blue{azul}, \green{verde}.}

\Dibujo[0.6]{Ejemplo de grafo $2$-partito.}

\begin{definition}
Un grafo $r$-partito $G$, donde cada par de vértices de partes distintas son adyacentes, decimos que $G$ es \textbf{$r$-partito completo}.

Un grafo $r$-partito completo con partes de tamaño $n_1,n_2,\ldots, n_r$ se denota $K_{n_1,n_2,\ldots, n_r}$.

\Dibujo[0.6]{Ejemplo de grafo $3$-partito completo.}
\end{definition}

\begin{obs}
Un grafo $K_{1,n}$ es una \textit{estrella}.

\Dibujo[0.60]{$K_{1,5}$}
\end{obs}

\begin{obs}
Si $G$ es bipartito, entonces no tiene ciclos impares.
\end{obs}
\begin{proof}
Sea $C=C_{2k+1}$ con $k \geq 1$ un subciclo de longitud $2k+1$ de $G$. Si $G$ fuera bipartito, entonces $C$ también. En efecto, numerando $C : x_0, x_1, \ldots, x_{2k} , x_0$, ser bipartito equivale a que existe una función $\rho : x_i \mapsto 0, 1 \in \{ 0 , 1 \}$ tal que $\rho (x) \neq \rho (y)$ para todo par de vértices adyacentes $x,y \in C$. Sin pérdida de generalidad $\rho (x_0 ) = 0$. Pero como $x_i$ y $x_{i+1}$ son siempre adyacentes, debe ser que $\rho (x_0 ) =  0 , \rho (x_1) = 1 , \ldots , \rho (x_i) = i \mod 2$ (lo podemos probar recursivamente). Con lo cual, $\rho ( x_{2k}) = 0 = \rho ( x_0)$, lo cual es absurdo porque $x_{2k}$ y $x_0$ son adyacentes.
\end{proof}


\Dibujo{Como se ilustra en el dibujo, no podemos $2$-particionar a $C_7,C_5$ ni $C_3$. Pues siempre que pintamos con dos colores quedan dos vértices adyacentes.}

\begin{Theorem}
Un grafo es bipartito si y solo si no tiene ciclos impares.
\end{Theorem}
\begin{proof}
La observación anterior prueba la necesidad. Veamos la suficiencia. Sea $G$ un grafo sin ciclos impares. Podemos asumir sin pérdida de generalidad que es conexo. Sea $T$ un árbol generador, $r$ un vértice de $G$ que llamaremos raíz (de $T$). Para $v \in V(G)$ denotamos por $rTv$ al único camino entre $r$ y $v$ en $T$ (por la caracterización de árboles). Por último, si $w,v \in V(G)$, entonces decimos que $w \leq v$, si $w \in rTv$.

Definimos la partición de $G$: los vértices $v$ tales que $rTv$ tiene largo par, y por otro lado los vértices $v$ tales que $rTv$ tiene largo impar. Veamos que en efecto esto es una partición, i.e., no hay vértices adyacentes en la misma partición. Sea $e = xy$ una arista de $G$.
\begin{enumerate}
\item[]

\item[\textsc{Caso 1:}] Si $e \in E(T)$, tendremos $x < y$ o $y < x$, pero nunca igualdad. Más aún, $\Abs{rTx} = \Abs{rTy} \pm 1$, i.e. tienen paridades distintas.

\item[\textsc{Caso 2:}]  Si $e \not \in E(T)$, entonces $rTx, rTy$ y $e$ forman un ciclo (por la caracterización de árbol). Por la hipótesis, el ciclo es par. Esto implica que $\Abs{rTx}$ y $\Abs{rTy}$ tienen distinta paridad:

\Dibujo[0.7]{Ilustración de este hecho.}
\end{enumerate}
\end{proof}

\begin{corollary}
Los árboles y los bosques son bipartitos.
\end{corollary}


\subsection{Paseos Eulerianos}

Viajamos a Prusia, siglo XVIII, a la ciudad de Königsberg.

\Dibujo{Los siete puentes de Königsberg.}

La gente de la ciudad se preguntaba si se podía partir de un punto $x \in A$ o $B$ de la ciudad, cruzar cada puente exactamente una sola vez y volver a $x$. Euler se propuso a responder la pregunta.

Podemos modelar el problema como un \textit{multi}grafo (i.e. dos vértices pueden estar unidos por más de una arista):

\Dibujo[0.7]{Multigrafo de los puentes de Königsberg.}


\begin{Definition}
Un \textbf{paseo} en un multigrafo, es una secuencia de vértices $x_0,x_1,\ldots, $ tal que $x_i x_{i+1}$ es arista para todo $i \geq 0$, y ninguna de estas aristas se repite.

Un \textbf{paseo cerrado} comienza y termina en el mismo vértice.

Un paseo es \textbf{Euleriano} si es cerrado y recorre todas las aristas del multigrafo.

Un \textbf{grafo Euleriano}, es un multigrafo que contiene un paseo Euleriano.
\end{Definition}

\begin{Theorem}
Un multigrafo conexo es Euleriano si y solo si todos sus vértices tienen grado par.
\end{Theorem}

Veremos la demostración la clase siguiente.












%%%%%%%%%%%
\Clase{30/03/23}   %%%%%%%%%%%%%%%%%%%%%%%%%%%%%%%%%%%%%%%%%%%%%%%%%%%%%%%%%%%%%
%%%%%%%%%%%



\begin{Theorem}
Un multigrafo conexo es Euleriano si y solo si todos sus vértices tienen grado par.
\end{Theorem}
\begin{proof}
\begin{enumerate}
\item[]


\item[$\Rightarrow$)] Asumimos que $G$ tiene un paseo Euleriano $P$. Cada vez que el paseo ''entra'' en una vértice, lo hace por medio de una arista, y debe salir por otra. Cada vez que $v$ a parece en $P$ se utilizan otras dos aristas incidentes en $v$. Como se ocupan todas esas aristas, $d(v)$ es par.

\item[$\Leftarrow$)]  Supongamos que todos los grados son pares. Haremos inducción en $\Abs G$. El caso base es $\Abs G = 2$ que claramente tiene un paseo Euleriano:

\Dibujo[0.1]{$\Abs G = 2$.}

Supongamos que $\Abs G >2$. Cuando todos los grados son pares, puedo encontrar un paseo cerrado no trivial. Tomemos como $P$ el de largo máximo, y sea $F$ su conjunto de aristas. Si $F$ es todo, la demostración está terminada. Luego supongamos que no. Sea $G \setminus F = G'$, tiene una arista $e$ que incide en un vértice de $P$, pues $G$ es conexo. Sea $C$ la componente de $G'$ que contiene a $e$. Todo vértice de $G$, posee un número par de aristas incidentes en $F$, luego la cantidad de aristas en $G'$ sigue siendo par. Aplicando la hipótesis inductiva, podemos encontrar un paseo Euleriano en $C$, llamémoslo $P'$. Como $P$ y $P'$ unidos son un paseo cerrado más grande que $P$, llegamos a un absurdo.

\Dibujo{Esta es una ilustración de lo que podría suceder: $P : 0,1,2,3,5,6,7,8,4,0$ es el camino \black{negro} y \darkgreen{$C$ es el grafo conexo}.}

\end{enumerate}

\end{proof}

\subsection{Conexidad}

\begin{Definition}
Decimos que un conjunto $X$ de vértices o aristas \textbf{separa} a $u,v \in V$ si $u,v \not \in X$ y todo camino entre $u$ y $v$ tiene un elemento de $X$.

Si $X$ separa un par de vértices, decimos que es \textbf{separador} (de $u,v$). Si un vértice solo, i.e. $X$ es un singleton, es separador, decimos que es un \textbf{vértice de corte}. Un pequeño abuso de notación será simplemente referirnos a ese vértice en lugar del conjunto que lo contiene.

Análogamente, una arista sola $X = \{e\}$ que separa sus vértices se dice \textbf{puente}. Un pequeño abuso de notación será simplemente referirnos a ese vértice en lugar del conjunto que lo contiene.
\end{Definition}

\Dibujo{Ejemplo: \yellow{$X = \{4,5,7,8,9 \}$} separa a \red{$u = 3,v = 6$}. También \purple{$Y = \{ e_1, e_2 \}$} e  \brown{$Y ' = \{ e_3, e_4\}$} son separadores de \red{$u,v$}. La arista \blue{$e = (6,10)$} es un puente.}

\begin{definition}
Para $k \geq 0$ decimos que $G = (V,E)$ es \textbf{$k$-conexo} si $\abs V > k$ y $G \setminus X$ es conexo para todo $X \subset V$ con $\abs X < k$. Es decir, ningún conjunto de menos de $k$-vértices separa.
\end{definition}

\begin{example}
\begin{itemize}
\item \textit{$0$-conexo:} Todo grafo no vacío.

\item \textit{$1$-conexo:} grafos conexos no triviales (tiene que tener al menos una arista).

\end{itemize}
\end{example}


\begin{definition}
La \textbf{conexidad}, $\Kappa (G)$ de $G$, es el máximo $k \geq 0$ tal que $G$ es $k$-conexo.
\end{definition}

\begin{example}\label{ex:clase6 - exemplo Kappa(K_n) = n-1}
\begin{itemize}
\item \textit{$\Kappa (G) = 0$:} Todo grafo disconexo no vacío o $K_1$.

\item \textit{$\Kappa (K_n) = n-1 , \forall n \geq 1$.}

\end{itemize}
\end{example}

\begin{definition}
Sea $G$ no vacío y sea $\ell \geq 1$. Decimos que $G$ es \textbf{$\ell$-arista conexo}, si $G \setminus F$ es conexo para todo $F \subset E$ con $\Abs F < \ell$.
\end{definition}

\begin{example}
\textit{$1$-arista conexo:} los grafos conexos no vacíos.
\end{example}

\begin{definition}
La \textbf{arista conexidad}, $\lambda (G)$ de $G$, es el máximo $\ell$ tal que $G$ es $\ell$-arista conexo.
\end{definition}

\Dibujo[0.8]{Ejemplo de grafo que tiene $\Kappa = 4, \lambda = 4$.}

\begin{example}\label{ex:clase6 - exemplo lambda (K_n) = Kappa(K_n) = n-1}
\textit{$\lambda (K_n) = n-1 , \forall n \geq 1$.}. Con lo cual, por el Ejemplo \ref{ex:clase6 - exemplo Kappa(K_n) = n-1} tenemos que 
$$
\lambda (K_n) = \Kappa (K_n) .
$$
\end{example}

\begin{exercise}
Calcular $\Kappa$ y $\lambda$ del siguiente grafo $G$:
\Dibujo[0.5]{Grafo $G$.}
\end{exercise}
\begin{solution}
Si quitamos los  \blue{vértices $4,5$} entonces nos queda $G$ disconexo, luego $\Kappa (G) <3$. Si quito cualquier vértice, entonces el grafo sigue siendo conexo, luego es $2$-conexo, i.e. $\Kappa (G) \geq 2$. Luego $\Kappa (G) = 2$.

En el anterior ejemplo teníamos $\lambda = 4$ en cada \blue{triángulo azul}, y como sacar $3$ aristas incidentes a \blue{$4,5$} no evita que $G$ siga siendo conexo, tenemos que $G$ es $4$-arista conexo, i.e. $\lambda (G) \geq 4$. Por otro lado, si quitamos las $4$ aristas incidentes en el \red{vértice $1$}, queda el aislado del resto del grafo, i.e. $\lambda (G) < 5$. Luego $\lambda (G) = 4$.
\end{solution}


%%%%%%%%%%%
\Clase{03/04/23}   %%%%%%%%%%%%%%%%%%%%%%%%%%%%%%%%%%%%%%%%%%%%%%%%%%%%%%%%%%%%%
%%%%%%%%%%%

\begin{Proposition}
Si $G$ es no trivial, entonces $\Kappa (G) \leq \lambda (G) \leq \delta (G)$
\end{Proposition}
\begin{proof}
La segunda desigualdad se tiene porque todas las aristas incidentes en un vértice fijo separan a $G$.

Veamos ahora la primera desigualdad. Sea $F$ un conjunto de aristas que separa a $G$, con $\abs F = \lambda (G)$, tal que $G \setminus F$ es disconexo. \textit{Observación:} $F$ es un conjunto de aristas minimal con la propiedad de ser separador.
\begin{enumerate}
\item[\textsc{Caso 1:}] Existe $v \in V(G)$ que no incide en $F$. Sea $C$ la componente conexa que contiene a $v$ en $G \setminus F$. No puede haber una arista $f$ de $F$ con extremos en $C$, pues de lo contrario $F \setminus \{f\}$ sería un conjunto más chico tal que es separador, lo cual contradice la minimalidad de $F$. 

\Dibujo[0.6]{Ilustración de \blue{$C$}, donde las \purple{aristas violeta} corresponden a aristas de \purple{$F$}.}

Luego, si quitamos los vértices de las aristas de $F$ incidentes en $C$, las cuales solo comparten un vértice de $C$, nos queda que $v$ estaría separado del resto del grafo. Esta cantidad de vértices es a lo sumo $\abs F = \lambda (G)$. Con lo cual $\Kappa (G) \leq \abs F = \lambda (G)$.

\item[\textsc{Caso 2:}] Todo $v \in V(G)$ incide en $F$. Fijemos $v \in V(G)$ y $C$ la componente conexa de $G \setminus F$ que lo contiene. Consideremos $N_{G} (v)$, los vecinos de $v$. Cada $w \in N_{G} (v) $ incide en una arista de $F$.

\Dibujo{Ilustración de lo que sucede: los \darkgreen{vecinos} de \green{$v$} inciden en una \purple{arista de $F$}.}

Entonces $d_G (v) \leq \abs F = \lambda (G)$. Por lo tanto, salvo que $V(G) = \{ v \} \cup N_G ( v)$, tenemos que $N_G (v)$ separa a $v$ del resto del grafo, y salvo ese caso tendríamos que $\Kappa (G) \leq \abs{N_G (v)} \leq \lambda (G)$. Pero $v$ era arbitrario, entonces en el peor de todos los casos, tenemos que $V(G) =  \{ v \} \cup N_G ( v)$ para todo $v \in G$, i.e. $G$ es un grafo completo. Pero en este caso vale la igualdad por el Ejemplo \ref{ex:clase6 - exemplo lambda (K_n) = Kappa(K_n) = n-1}.
\end{enumerate}
\end{proof}

\subsection{Grafos $2$-conexos}

\begin{Definition}
Sea $H$ un grafo. Decimos que un camino $P$ es un \textbf{$H$-camino} si es no trivial (tiene al menos una arista) e interseca a $H$ exactamente en sus extremos ($P$ no tiene ni vértices ni aristas en $H$, salvo por sus extremos).
\end{Definition}

\Dibujo[.7]{Ejemplo de \blue{$H$}-camino \yellow{$P$} de un grafo \blue{$H$}. Notar que en el dibujo consideramos a los vértices \blue{$0,5$} como extremos de \yellow{$P$}.}

Los ciclos son los grafos $2$-conexos más elementales. Veamos que todos los demás se pueden construir a partir de ellos.

\Dibujo[.7]{Ejemplos de ciclos: $C_7, C_6, C_5, C_4$ y $C_3$.}

\begin{proposition}
Un grafo es $2$-conexo si y solo si se puede construir a partir de un ciclo añadiendo sucesivamente $H$-caminos a grafos $H$ ya construidos.
\end{proposition}

\begin{remark}
Es decir, si $H_0$ es un ciclo, le agregamos un $H_0$-camino, y a la unión la llamamos $H_1$, el cual es $2$-conexo, si quisieramos podemos agregar un $H_1$-camino y seguiría siendo $2$-conexo, etc.
\end{remark}

\Dibujo{Ilustración de un ciclo $H_0$ en negro, al que le agregamo $H$-caminos en el siguiente orden: \yellow{$H_0$-camino}, \red{$H_1$-camino} y por último \blue{$H_2$-camino}.}

\begin{proof}
\begin{enumerate}
\item[]

\item[ ( $\Leftarrow$ )] Claramente un grafo construido de esta manera no se puede separar por un solo vértice. Y por su puesto que tiene más de $2$ vértices.

\item[ ($\Rightarrow$)] Tomemos un grafo $G$, $2$-conexo (en particular es también conexo). Como es $2$-conexo, debe tener algún ciclo $C$, pues de lo contrario sería un árbol con al menos $3$-vértices, y quitando un vértice que no es hoja nos quedaría separado. Nos fijamos si tiene un $C$-camino, si esto es así lo agregamos, y luego seguimos agregando hasta que no podamos más. Consideremos el subgrafo maximal $H$ de $G$ construido de esta manera a partir de $C$. Entonces toda arista $xy \in E(G) \setminus E(H)$ tal que $x,y \in V(H)$, es un $H$-camino, con lo cual no puede existir por maximalidad de $H$. Es decir, $H$ es un subgrafo inducido de $G$. Entonces todas las aristas de $G$ que no están en $H$ tienen un extremo fuera de $H$. Si $H$ es $G$ habríamos terminado, luego por el absurdo supongamos que no. Por conexión existe un vértice $v \in G \setminus H$, y conectándolo por un camino con $H$ podemos asumir que $v$ es incidente en $H$, es decir existe $w \in H$ tal que $v w$ es una arista incidente en $H$. Como $G$ es $2$-conexo, si quitamos a $w$ el grafo sigue siendo conexo, luego debe ser que existe otro camino $P$ de $v$ a $H$, con lo cual $P \cup v w$ es un $H$-camino. Absurdo por maximalidad de $H$.
\end{enumerate}
\end{proof}


\Dibujo{Ilustración: \green{$H$} contiene a \green{$w$}, con una arista incidente de extremo \blue{$v$}, que se extiende a un \green{$H$}-camino \blue{P}.}







%%%%%%%%%%%
\Clase{06/04/23}   %%%%%%%%%%%%%%%%%%%%%%%%%%%%%%%%%%%%%%%%%%%%%%%%%%%%%%%%%%%%%
%%%%%%%%%%%




Todo grafo \textit{sin vértices aislados} se puede particionar en subgrafos $1$-conexos. Y podemos intentar lo mismo para subgrafos $2$-conexos. Pero pueden ocurrir problemas, por ejemplo:

\Dibujo{Ejemplo de problemas para particionar en subgrafos $2$-conexos maximales. Las componentes $2$-conexas del dibujo son sus \blue{ciclos} que comparten vértices con otras estructuras como por ejemplo las dos \yellow{aristas} $12$ y $23$; o comparten aristas entre dos ciclos.}

Como ilustra la figura de arriba, los subgrafos $2$-conexos maximales no siempre abarcan todo el grafo ni son siempre disjuntos. Veamos como se arregla: podemos simplificar la noción para poder abarcar todo el grafo.

\begin{definition}
Un \textbf{bloque} es un subgrafo conexo maximal sin vértices de corte.
\end{definition}

En la figura anterior, los bloques del grafo con los \blue{ciclos} y las \yellow{aristas} $12$ y $23$.

\begin{obs}
Es fácil ver que los bloques van a ser o subgrafos \textit{$2$-conexos} o una \textit{arista} o un \textit{vértice}.
\end{obs}

\begin{proposition}
Los ciclos de un grafo son los ciclos de sus bloques.
\end{proposition}
\begin{proof}
Todo ciclo es $2$-conexo, luego es conexo sin vértices de corte, y debe estar contenido en un subgrafo maximal con esta propiedad, i.e. un bloque.
\end{proof}

\begin{proposition}
Sean $e,f \in E(G)$. Entonces pertenecen a un mismo bloque si y solo si pertenecen a un mismo ciclo.
\end{proposition}
\begin{proof}
Si pertenecen al mismo ciclo, entonces por la proposición anterior están en el mismo bloque.

Recíprocamente, como $e,f$ son dos aristas en un mismo bloque, puedo asumir que el bloque es un subgrafo $2$-conexo (no es arista sola o vértice solo). La idea es la siguiente: este subgrafo $2$-conexo se construye a partir de un ciclo uniendo $H$-caminos, luego no es difícil ver que las dos aristas están contenidas en un mismo ciclo.
\end{proof}

\begin{definition}
El \textbf{grafo bloque} de un grafo $G$ tiene un vértice por cada bloque y por cada vértice de corte de $G$; hay una arista entre dos vértices si una representa un bloque y si el otro representa un vértice de corte que está dentro del bloque.
\end{definition}

\Dibujo{Ejemplo: $G$ (izquierda) tiene $5$ \purple{bloques} (denotados por letras mayúscula: \purple{$A,B,C,D,E$}) y $2$ \darkgreen{puntos de corte} (denotados por letras minúscula: \darkgreen{$a,b$}). Luego el grafo bloque (derecha) de $G$ tiene $7$ vértices.}

Notar que en nuestro ejemplo, el grafo bloque es un árbol. Esto no es casualidad:

\begin{exercise}
El grafo bloque de un grafo conexo es un árbol.
\end{exercise}
\begin{proof}
\textit{Notación:} al grafo bloque de $G$ lo denotamos por $Block(G)$. A un subgrafo conexo sin vértices de corte
maximal,
i.e. un bloque, lo vamos a denotar con las letras mayusculas $B,C,D$. Y denotaremos con la misma letra al vértice que inducen en el grafo $Block(G)$. A los vértices de corte los denotaremos por una letra minúscula como $x,y,z,u,v,w$ y los denotaremos de la misma manera en el grafo $Block (G)$. Quedará claro dependiendo del contexto, a qué grafo pertenece cada vértice en esta notación. Haremos el abuso de notación y llamaremos bloque tanto al subgrafo de $G$ como al vértice de $Block (G)$. Análogamente, cuando digamos vértice de corte de $Block(G)$ nos estamos refiriendo a un vértice que proviene de un vértice de corte de $G$.

Si $G$ es conexo, luego $Block(G)$ es conexo. Antes notemos que basta probar que entre dos blockes de $Block (G)$ existe un camino, pues todo vértice de corte de $Block (G)$ es adyacente a algún bloque en $Block (G)$ por definición de grafo bloque. Sean $B,B'$ dos bloques de $Block(G)$, consideremos luego a partir de un $B,B'$-camino siempre podemos construir un camino que no puede entrara y salir de un bloque más de una vez, por conexión del bloque. Este camino nos induce un camino en $Block(G)$ dado por $\tilde P : B_0 v_0 B_1 v_1 \cdots B_{r-1} v_{r-1} B_r $, donde cada bloque o vértice aparece en el orden en el cual el camino $P$ se intersecó por primera véz con estos en $G$.

Ahora vevamos que $Block(G)$ es aciclico. En efecto, supongamos que no, sea $C$ un ciclo en $Block(G)$. Como $Block(G)$ es bipartito (particionamos entre vértices de corte y bloques), no tiene ciclos impares, luego $C$ tiene al menos $4$ vértices (pueden ser cortes o bloques). Con lo cual, existen dos bloques distintos $B_1,B_2$ y dos vértices de corte distintos $v_1,v_2$ tal que podemos escribir $C : B_1 v_1 B_2 \cdots v_2 B_1$. Pero esto quiere decir que hay otro $B_1,B_2$-camino en $G$ que no pasa por $v_1$, es decir que $v_1$ no era vértice de corte, absurdo.
\end{proof}


\subsection{Contracciones y menores}

\begin{definition}
\textbf{Contraer una arista} $e=xy$ equivale a borrar $x$ e $y$, y añadir un nuevo vértice $v_{xy}$ adyacente a todos los vértices que eran vecinos a $x$ o $y$.
\end{definition}

\Dibujo{Ejemplo. Contraemos los vértices \blue{$x,y$} y formamos $\blue{v_{xy}}$.}

\begin{notation}
Dado un grafo $G$ y $e = xy \in E(G)$, notamos como $G/e$ al grafo que se obtiene de $G$ al contraer la arista $e$.
\end{notation}

\begin{definition}
Decimos que $H$ es un \textbf{menor} de $G$ si se puede obtener $H$ a partir de $G$ al utilizar las siguientes operaciones:
\begin{enumerate}[1.]
\item Borrar vértices.
\item Borrar vértices y aristas.
\item Contraer aristas. O equivalentemente, contraer subgrafos conexos.
\end{enumerate}
\end{definition}

\begin{example}
Los subgrafos y contracciones de $G$ son \textit{menores} de $G$. No necesariamente vale la vuelta:

\Dibujo{Ejemplo de menor \blue{$H$} de \green{$G$}, que no es subgrafo porque tiene grado máximo $\Delta (H) = 5$. Pues \blue{$H$} se obtiene luego de contraer las \yellow{aristas} de \green{$G$}}
\end{example}


\subsection{subdivisiones}

Sea $X$ un grafo fijo.

\begin{definition}
Llamamos \textbf{subdivisión} de $X$ a cualquier grafo $G$ que se obtiene de \textit{subdividir} algunas aristas de $X$ y dibujando encima nuevos vértices. Más precisamente, reemplazamos las aristas de $X$ con nuevos caminos entre sus extremos, de manera que estos caminos no se intersecan entre si y tampoco intersecan a $V(X)$ salvo en los extremos.
Diremos que $G$ es un $TX$.

Llamaremos a los vértices originales de $X$, \textbf{vértices de ramificación de los de} $TX$; a los nuevos vértices los llamaremos \textbf{vértices subdivisores}.

Si un grafo $Y$ contiene a $TX$ como subgrafo, diremos que $X$ es una \textbf{menor topológica} de $Y$.
\end{definition}
Notar que los vértices subdivisores tienen grado $2$ y los vértices de ramificación no cambian de grado.

\Dibujo{De izquierda a derecha, tenemos la construcción prograsia de $X$, luego le agregamos \red{vértices subdivisores} formando $TX$, y finalmente ilustramos un ejemplo de grafo $Y$ con $X$ como menor topológico.}



\begin{definition}
Similarmente, reemplazando los vértices $x \in X$ con grafos conexos disjuntos $G_x$, y las aristas $xy \in X$ con conjuntos no vacíos de $G_x - G_y$ aristas, obtenemos un grafo que llamaremos $IX$. Recíprocamente, decimos que $X$ se obtiene a partir de $G$ \textbf{contrayendo} subgrafos $G_x$ (y fusionando las $G_x-G_y$ aristas), y lo llamamos una \textbf{menor contraida} de $G$.

Si un grafo $Y$ contiene un $IX$ como subgrafo, decimos entonces que $X$ es una \textbf{menor} de $Y$, llamamos a $IX$ un \textbf{modelo} de $X$ en $Y$, y denotamos $X\preccurlyeq Y$
\end{definition}

\Dibujo{De izquierda a derecha, tenemos el grafo $X$, que se obtiene de contraer \purple{sub}\red{gr}\darkgreen{af}\blue{os} de $G$; finalmente, el grafo $Y$ tiene a $X$ como menor y a $G$ como modelo de $X$ en $Y$}

Por lo tanto, $X$ es un menor de $Y$ si y solo si existe una función $\varphi : S \subset V(Y) \twoheadrightarrow V(X)$ tal que para todo vértice $x \in X$ si preimagen $\varphi^{-1} (x)$ es conexa en $Y$ y para toda arista $xx' \in E(X)$ existe una arista en $Y$ entre conjuntos de ramificación $\varphi^{-1}(x), \varphi^{-1} (x')$. Si el dominio de $\varphi$ es todo $V(Y)= S$, y si $xx' \in E(X)$ siempre que $x \neq x'$ e $Y$ tiene una arista entre $\varphi^{-1}(x)$ y $\varphi^{-1} (x')$ (es decir $Y$ es una $IX$), decimos que $\varphi$ es una \textbf{contracción} de $Y$ en $X$.


\begin{proposition}
La relación de menores $\preccurlyeq$ y la relación de menores topológicos son ordenes parciales en la clase de grafos finitos. Es decir, son reflexivos, antisimétricos, y transitivos.
\end{proposition}


Si $G$ es una $IX$, luego $P = \Set{G_x | x \in X}$ es una partición de $V(G)$, y notamos $G/P := X$. Si $U = G_x$ es el único conjunto de ramificación que no es un singleton, escribimos $G/U := X$, y notamos $v_U$ al vértice $x \in X$ al que se contrae $U$, y pensamos al ressto de $X$ como un subgrafo inducido de $G$. El caso más simple es cuando $U$ contiene exactamente dos vértices que forman una arista $e= U$, aquí escribiremos $G/e = X$, el grafo que se obtiene de \textbf{contraer la arista} $e$.

\begin{proposition}
Sean $X$ e $Y$ grafos finitos. Entonces $X$ es una menor de $Y$ si y solo si existen grafos $G_0,\ldots,G_n$ tales que $G_0 = Y$ y $G_n = X$, y además $G_{i+1}$ se obtiene a partir de $G_i$ borrando aristas, contrayendo aristas, o borrando vérticces.
\end{proposition}
\begin{proof}
Estas tres últimas operaciones claramente producen una menor $X$, pues la relación de menor es transitiva. Recíprocamente, se puede hacer inducción en $\abs Y + \Abs Y$.
\end{proof}

Finalmente, tenemos la siguiente relación entre menores y menores topológicos:

\begin{proposition}
\begin{enumerate}[(i)]
\item Todo $TX$ es también un $IX$ (ver la siguiente figura); por lo tanto, toda menor topológica de un grafo es su menor (ordinaria).
\item Si $\Delta (X) \leq 3$, entonces todo $IX$ contiene un $TX$; con lo cual, toda menor con grado máximo a lo sumo $3$ de un grafo es también su menor topológico.
\end{enumerate}
\end{proposition}
\begin{proof}
Veamos solo (ii), el primer ítem es obvio. En efecto, $IX$ es el grafo que se obtiene de $X$ reemplazando cada vértice $x$ de él por un subgrafo conexo $G_x$ y cada arista por un conjunto de aristas no vacío, luego tomando una arista de ese conjunto, basta con escoger un vértice de $G_x$ que tenga por cada vecino de $x$ en $X$ un camino distinto hacia cada arista incidente en $G_x$. Esto es posible: empezamos eligiendo de manera inocente al vértice que es extremo de una arista incidente con $G_x$, de este vértice, llamemosló $x$, tendríamos que contrar dos caminos disjuntos con extremo final, en el peor de los casos pues $\Delta (X) \leq 3$, en otras dos aristas de $X$; ahora si no se cruzan ya ganamos, de lo contrario se fusionan a partir de un momento, incluso varias veces, pero luego movemos nuestro vértice $x$ a la última véz que se fusionan los caminos, y llamemosló $x'$:

\Dibujo{Ilustración de cómo se ven dos caminos, ambos de color amarillo pero uno más claro que el otro, que salen de nuestro vértice $x$ y tienen que llegar a las aristas incidentes en $G_x$. Al final movemos nuestro vértice a $x'$.}

\end{proof}

\Dibujo{Ejemplo: Una subdivisión de $K^4$ visto como $I K^4$.}


Ahora que conocemos todas las relaciones standard entre grafos, podemos definir lo qu esignifica embeber a un grafo en otro.

\begin{definition}
Básicamente, una \textbf{inmersión} (o \textbf{embedding} en inglés) de $H$ en $G$ es un mapa inyectivo $\varphi : V(H) \rightarrow V(G)$ tal que preserva la estructura en la que estamos interesados. Con lo cual, $\varphi$ embebe a $H$ en $G$ c omo un subgrafo si preserva la adyacencia entre vértices, y como subgrafo inducido si preserva tanto la adyacencia como la no adyacencia. Si $\varphi$ está definido también en $E(H)$ como en $V(H)$ y manda $xy$ en caminos independientes de $G$ entre $\varphi (x)$ y $\varphi (y)$, decimos que $\varphi$ embebe a $H$ en $G$ como un menor topológico. Similarmente, decimos que es una inmesión de $H$ en $G$ como un menor, si mapea a $V(H)$ en conjuntos disjuntos de vértices en $G$ conexos, de manera que $G$ tiene una arista entre los conjuntos $\varphi (x)$ y $\varphi (y)$ siempre que $xy$ es una arisda de $H$. Más varíantes existen, pero depende del contexto en el que estemos; por ejemplo, se pueden definir de manera obvia las inmersiones de 'subgrafos generadores', y 'menores inducidas', etc.
\end{definition}

%%%%%%%%%%%
\Clase{17/04/23}   %%%%%%%%%%%%%%%%%%%%%%%%%%%%%%%%%%%%%%%%%%%%%%%%%%%%%%%%%%%%%

%%%%%%%%%%%


\begin{lemma}
Todo grafo $3$-conexo, distinto de $K_4$, tiene una arista $e$ tal que $G/e$ es $3$-conexo.
\end{lemma}
\begin{proof}
Supongamos que no hay una arista con esta propiedad. Es decir, para toda $xy \in E(G)$, el grafo $G/xy$ tiene un conjunto separador $S$ de a lo más $2$ vértices. Como $G$ es $3$-conexo, $v_{xy}$ (el vértice que sale de contraer la arista $xy$) tiene que estar en $S$ y $\abs S = 2$, ya que $\abs S = 0$ es imposible porque el grafo es conexo y por otro lado si $\abs S = 1$, estamos diciendo que $G$ se puede separar con un solo vértice si $v_{xy} \not \in S$ o si $v_{xy}\in S$ entonces podemos desconectar a $G$ con los vértices $x, y$, imposible.
Luego, escribimos $S = \{z, v_{xy}\}$ con $z \not \in \{x,y\}$ separador de $G/xy$. Con lo cual, $T=\{z,x,y\}$ separa a $G$. Como ningún subconjunto propio de $T$ separa, cada vértice de $T$ tiene un vecino en cada componente de $G \setminus T$.
\Dibujo{Ilustración del conjunto \purple{$T$}, donde se muestra que el vértice \purple{$z$} tiene vecinos en cada \yellow{com}\red{po}\blue{nente} de $G \setminus T$.}

Como $xy$ era arbitrario, podemos elegir la arista $xy$, el vértice $z$ y la componente $C$ tal que $\abs C$ es mínimo. Tomo $v \in C$ y es vecino de $z$. Entoncces $G/v z$ tampoco es $3$-conexo, o sea que existe $w$ tal que $v,z,w$ (distintos) separan $G$. Como antes cada una de $v,z,w$ tiene un vecino en cada conmponente de $G \setminus \{v,z,w\}$. Como $x$ e $y$ son adyacentes, existe $D$ componente de $G \setminus \{v,z,w\}$ tal que $D \cap \{x,y\} = \emptyset$ (porque $G \neq K_4$!).
\Dibujo{\blue{$D$} está contenido en el \blue{subgrafo azul}.}
Dado que $v \in C$, los vecinos de $v$ en $D$ están en $C$. Tenemos que $D \cap C \neq \emptyset$, más aún, $D \subsetneq C$. Contradiciendo la minimalidad del orden de $C$.
\end{proof}

\begin{theorem}[Teorema de Tutte, 1961]
Un grafo $G$ es $3$-conexo si y solo si existe una secuencia de grafos $G_0, G_1, \ldots, G_n$ que cumple lo siguiente:
\begin{enumerate}[(i)]
\item $G_0 = K_4$ y $G_n = G$.
\item $G_{i+1}$ tiene una arista $xy$ tal que $d(x),d(y) \geq 3$ y $G_i = G_{i+1}/xy$ para todo $i < n$. Más aún, cada $G_i$ es $3$-conexo.
\end{enumerate}
\end{theorem}
\begin{proof}
Por el lema anterior, podemos quitar una arista recursivamente hasta llegar a $K_4$. La vuelta esta en el DIESTEL.
\end{proof}


\subsection[]{Teorema de Menger}
\begin{definition}
Si $A,B,X \subset V(G)$ son tales que todo $A,B$-camino tiene un vértice de $X$, decimos que $X$ separa a $A$ y $B$ en $G$.
\end{definition}

\begin{theorem}[Menger, 1927]
Sea $G = (V,E)$ un grafo y sean $A,B \subset V$. El mínimo número de vértices que separa a $A$ y $B$ es igual al máximo número de $A,B$-caminos disjuntos.
\end{theorem}







%%%%%%%%%%%
\Clase{20/04/23}   %%%%%%%%%%%%%%%%%%%%%%%%%%%%%%%%%%%%%%%%%%%%%%%%%%%%%%%%%%%%%
%%%%%%%%%%%



\begin{theorem}[Menger, 1927]
Sea $G = (V,E)$ un grafo y sean $A,B \subset V$. El mínimo número de vértices que separa a $A$ y $B$ es igual al máximo número de $A,B$-caminos disjuntos.
\end{theorem}
\begin{proof}
Sea $k$ el mínimo numero de vértices que separan $A$ y $B$. No es difícil convencerse que $G$ no puede contener más de $k$ caminos entre $A$ y $B$, i.e. $k\geq$ al máximo número de $A,B$-caminos disjuntos.

\Dibujo{El \brown{conjunto separador con $k$-elementos} se ilustra en el medio de los conjuntos \red{$A$} y \blue{$B$}, formados por los vértices \brown{$0$},\red{1},\red{2},\red{3} y \brown{$0$},\blue{4},\blue{5},\blue{6} respectivamente.}

Para la otra desigualdad, haremos inducción en el número de aristas. Si $G$ no tiene aristas entonces los $A,B$-caminos son puntos de $A \cap B$! y trivialmente vale la igualdad. Ahora, si existe una arista $e = xy$ de $G$, y si $G$ no tiene $k$ caminos entre $A,B$ disjuntos (es decir tiene $<k$), entonces $G/e$ tampoco pues $G/e$ no puede tener más $A,B$-caminos (contamos a $v_e$ como elemento de $A$ (o $B$) si alguna de $x $ o $y$ está en $A$ (o $B$)). Luego por hipótesis inductiva, $G/e$ tiene un $A,B$-separador $Y$ con menos de $k$ vértices. El vértice $v_e$ debe estar en $Y$, porque si no $Y$ sería separador de $G$, contradiciendo minimalidad de $k$. Entoncces $X = (Y \setminus v_e) \cup \{x,y\}$ es un $A,B$-separador de $G$ con exactamente $k$ vértices. En efedcto, por minimalidad $k \leq \abs X$, y por construccón $\abs X = \abs Y + 1 < k +1 \leq k$.

Consideremos ahora $G \setminus e$. Como $x,y \in X$, todo $A,X$-separador en $G \setminus e$ es un $A,B$-separador en $G$ con al menos $k$-vértices por minimalildad de $k$.

\Dibujo{Ilustración del conjunto separador \brown{$X$} y los conjuntos de vértices \red{$A$} y \blue{$B$}. Notar que todos los $A,B$-caminos deben pasar por el \yellow{$A,X$-separador}.}

Por inducción, hay al menos $k$ caminos entre $A,X$ disjuntos en $G \setminus e$. Lo mismo pasa con los $B,X$-caminos. Como $X$ separa a $A$ y $B$, estos caminos solo se encuentran en $X$ y los puedo combinar para tener al menos $k$ caminos entre $A$ y $B$ (disjuntos), contradicción.
\end{proof}

\begin{definition}
El \textbf{grafo línea} $L(G)$ de un grafo $G = (V,E)$ es aquel cuyo conjunto de vértices es $E$ y hay una arista entre dos elementos de $E$ si y solo si las aristas son adyacentes en $G$, es decir comparten un extremo.
\end{definition}

\Dibujo{Ejemplo de un grafo $G$.}

\Dibujo{Grafo de línea de $G$, donde se puede ver que sus \brown{vértices} son las \brown{aristas} de $G$.}


\begin{definition}
Un conjunto de $a,B$-caminos es un \textbf{$a,B$-abanico} si cada par de estos caminos se intersecta SOLO en $a$.
\end{definition}

\Dibujo{Ejemplo de $\red{a},\blue{B}$-abanico.}


\begin{corollary}
Para $B \subset V$ y $a \in V \setminus B$, el mínimo número de vértice que separan $a$ de $B$ en $G$ es igual al máximo número de caminos en un $a,B$-abanico en $G$.
\end{corollary}
\begin{proof}
Aplicamos el Teorema de Menger a $G \setminus a$ con conjuntos $A = N_G (a)$ y $B$ como en el enunciado.

\Dibujo{Ilustración del procedimiento: en rojo los \red{vecinos} de $a$, en azúl el conjunto \blue{B}, y en marrón a un connjunto $\red{A},\blue{B}$-separador \brown{$X$}.}

\end{proof}


\begin{corollary}
Sean $a$ y $b$ vértices distintos de $G= (V,E)$.
\begin{enumerate}[(i)]
\item Si $ab \not \in E$ ($a,b$ no son adyacentes), entonces el mínimo número de vértices que separan $a$ de $b$ en $G$ es igual al máximo número de $a,b$-caminos INTERNAMENTE disjuntos.
\item El mínimo número de aritas que separan $a$ de $b$ es igual al máximo número de $a,b$-caminos arista-disjuntos.
\end{enumerate}
\end{corollary}
\begin{proof}
\begin{enumerate}[(i)]
\item Aplicamos el Teorema de Menger a $G \setminus \{a,b\}$ con $A = N_G (a)$ y $B = N_G (b)$.
\item Aplicamos el Teorema de Menger al grafo $L(G)$ con conjuntos $A = E(a)$ y $B = E(b)$ los conjuntos de aristas incidentes en $a$ y $b$, respectivamente.
\end{enumerate}
\end{proof}

Notar que en el segundo ítem de la demostración anterior usamos que hay una correspondencia biyectiva entre aristas $A,B$-separadoras y vérticces separadores de $E(A),E(B)$ en $L(G)$, y también entre los $A,B$-caminos arista-disjuntos y los $E(A),E(B)$-caminos disjuntos de $L(G)$.

\begin{theorem}[Versión global de Menger]
\begin{enumerate}[(i)]
\item Un grafo es $k$-conexo si y solo si contiene $k$-caminos internamente disjuntos entre cada par de vértices.
\item Un grafo es $k$-aristaconexo si y solo si contiene $k$ caminos arista disjuntos entre cada par de vértices.
\end{enumerate}
\end{theorem}
\begin{proof}

\end{proof}





\section{Ejercicios}

\begin{exercise}
Sea $G$ un grafo que contiene un ciclo $C$, y supongamos que $G$ contiene un camino de longitud al menos $k$ entre
dos vértices de $C$. Probar que $G$ contiene un ciclo de longitud al menos $\sqrt k$.
\end{exercise}
\begin{solution}
    Si $C$ tiene longitud $\sqrt k$ entonces la afirmación vale. Si no, denotemos por $P$ al camino de longitud $k$
    entre dos vérticces $x,y \in C$. Como $\Abs C < \sqrt k$, $P$ interseca con $C$ en menos de $\sqrt k$ vértices,
    por lo tanto existen dos vértices $a,b \in P \cap C$ tales que, en el orden inducido por el camino $P$, no hay otro
    vértice de $C$ entre estos, y $a P b$ tiene longitud $\geq \sqrt k$. Luego el ciclo $a P b C a$ tiene logitd $\geq \sqrt k$.
\end{solution}


\begin{exercise}
Probar que los grafos de cintura $\geq 5$ y orden $n$ tienen $\delta = o (n)$. Es decir, existe $f : \naturals \rightarrow \naturals$ tal que $f(n) /n \rightarrow 0$ cuando $n \rightarrow \infty$ y $\delta (G) \leq f(n)$ para todo $G$ de orden $n$.
\end{exercise}
\begin{solution}
    En efecto, tenemos que
    $$
    n = \abs G \geq n_0 ( \delta, 5) = 1 + \delta (1 + (\delta -1)) = 1 + \delta^2
    $$
    por el Teorema débil \ref{}, si $\delta \geq 2$.
\end{solution}

\begin{exercise}
Probar que todo grafo conexo $G$ contiene un camino o un ciclo de longitud al menos $\min \{2 \delta (G) , \abs G\}$.
\end{exercise}
\begin{solution}
Comentario, el resultado es falso solo para $n = 2$. En efecto, veremos en la demostración que vale para $n \neq 2$,
y un camino de longitud uno no cumple pues $2 \delta = 2$ y $n = 2$ pero la longitud de cualquier sub camino o ciclo
es a lo más uno.

Procederemos por inducción en $\abs G = n$. Vale trivialmente para $n=1$; vimos que para $n=2$ es falso; pero para $n
=3$ vale pues $G$ es un camino o un ciclo de longitud $3$ y aquí si vale la afirmación. En general, supongamos que es
 falso para algún $n \geq 4$, el cual podemos tomarlo mínimo. Sea $P$ un camino o ciclo de longitud máxima $m$ en $G$, notar que $m < 2 \delta (G), n$. Supongamos que $P$ es un camino, luego sus extremos no pueden ser adyacentes (por maximalidad de $P$), pero tampoco pueden tener tener vecinos

SEGUIR PENSANDO
 Sea $v \in G \setminus P$. Debe ser que todos sus vecinos son vértices de $P$. En efecto, completar...
 Luego, notar que si $x,y \in N_G(v)$ son distintos, entonces no pueden ser adyacentes por maximalidad de $P$, luego $P$ tiene logitud $ \geq 2 (\delta(G))$ si $v$ no tiene grado mínimo, con lo cual todos los vértices fuera de $P$ tienen grado $\delta(G)$ y además $P$ tiene longitud $\geq 2 (\delta(G)-1)$.

 Si $P$ es un ciclo la demostración es una versión más facil de la anterior y es análoga.

 Si $P$



\end{solution}

\begin{exercise}
Sean $\alpha, \beta$ dos invariantes de grafos en $\naturals$.
\end{exercise}

\begin{exercise}
Probar que todo árbol $T$ tiene al menos $\Delta (T)$ hojas.
\end{exercise}
\begin{proof}
En efecto, fijemos una raíz $r$ con $d (r) = \Delta (t)$. Afirmamos que hay una hoja distinta por cada vecinos de $r$, más aún, estos son los elementos maximales en el orden de arbol con raíz $r$. Y hay al menos $d(r)$ de estos, tomand el máximo de cada conjuntos $w^{\geq} := \{ v \in T | v \geq w \}$, con $w \in N_T (r)$. Son distintos, pues de lo contrario, sean $m_1$ al mínimo vértice de $w_1^{\geq}$ y $m_2$ el de $w_2^{\geq}$ tales que no son más grandes que $w_2$ y $w_1$ respectivamente. En particular, existe $m \geq w_1,w_2$. Luego tenemos un ciclo $r <  w_1 < \cdots < m_1 < m > m_2 > \cdots > w_2 > r$, lo cual es imposible.
\end{proof}




\begin{exercise}
Sean $F,F'$ dos bosques en el mismo conjunto de vértices, y $\Abs F < \Abs  {F'}$. Probar que $F'$ tiene una arista $
e$ tal que $F + e$ es nuevamente un bosque.
\end{exercise}
\begin{solution}
    En efecto, si $F$ tuviera más de una componente, entonces cualquier arista $e \in F'$ con extremo en ambas
    funcionaría. Luego supongamos que $F'$ no tiene aristas que conectan ningúna componente de $F$, es decir por
    inducción en $\Abs F$ se sigue el resultado. Luego supongamos que $F$ es conexo, es decir es un árbol, como $\Abs {F'} = \abs {F'}- \# \text{componentes}$, se sigue de la desigualdad del enunciado que $F'$ también es conexo. Ahora si consideramos el grafo $G = F + F$, este grafo tiene dos árboles generadores $F$ y $F'$. Si $F$ y $F'$ tuvieran una arista en común, luego el resultado se sigue. De lo contrario, tendríamos que por cada arista $e \in F'$, tenemos un ciclo en $T+e$, es decir un ciclo fundamental $C_e$. Pero por la Proposición \ref{}:
    \[
    \Abs F + \Abs {F'} = \dim \mathcal E (G) = n -1 + \dim \mathcal C (G)
    \]
    con $n = \abs F = \abs {F'}$, y como $F$ es árbol $n-1 = \Abs F$. Luego $\dim \mathcal C (G) = \Abs {F'}$. An
    álogamente, $\dim \mathcal C (G) = \Abs {F} < \Abs {F'}$, imposible.
\end{solution}

\begin{exercise}
    Probar que todo grafo es $2$-arista-conexo, si y solo si, tiene una orientación \textbf{fuertemente conexa}, es
    decir, tiene una orientación en la cual para todo par de vértices $x,y$ existe un camino dirigido $\overset{
    \rightarrow}{P}$ con dirección de $x$ hacia $y$.
\end{exercise}
\begin{solution}
    \begin{enumerate}
    \item[($\Rightarrow$)] Numeremos los vértices de $G$: $v_{1},v_2, v_3,\ldots, v_n$
    \item[($\Leftarrow$)]
    \end{enumerate}
\end{solution}

\begin{exercise}
    Dar una demostración corta por inducción de la exisstencia de un árbol normal generador en cualquier grafo finito
     conexo (para cualquier orden de árbol).
\end{exercise}
\begin{solution}
    Afirmamos que existe un vértice $v$ de $G$ que se puede eliminar y sigue siendo conexo: $G$ tiene un árbol generador
    , luego quitamos una hoja. Ahora por inducción, $G \setminus v$ tiene un árbol generador normal $T'$. Afirmamos
    que el árbol generador $T = T' vx$ de $G$ es también normal, donde $x \in T'$ es adyacente a $v$ en $G$, maximal
    en el orden de $T'$.
    Ahora, el orden de $T'$ se extiende al de $T$ para cualquier raíz $r$ de $T'$. Sea $P$ un $T$-camino entre dos vértices
    distintos de $v$, luego son comparables en $T'$; por otro lado, si uno de los vértices es $v$ y el otro es $y$,
    digamos, entonces $x v P y$ es un $T'$-camino, luego $x$ e $y$ son comparables, y por lo tanto $v$ e $y$ tambi
    én, pues por la maximalidad de $x$, $x \geq y$. (
    Notar
    que
    luego
    vale para $v$
    como raíz
    tambi
    én.)
\end{solution}


\begin{exercise}[Depth-first search]
Sea $G$ un grafo conexo, y $r \in G$ un vértice arbitrario. Empezando desde $r$, nos movemos a través de las aristas
de $G$, priorizando movernos a un vértice que no hayamos visitado aún. Si no hay ningún vértice, retrocedemos por las
 aristas que visitamos por última véz, ordenadamente: la más reciente primero, intentando ocupar un vértice no
 visitadao nuevamente. El algorítmo para cuando regresamos a $r$. Probar que las aristas recorridas forman un árbol
 normal generadores de $G$ con raíz $r$.
\end{exercise}
 \begin{solution}
 Debemos probar varias cosas, primero que este recorrido, que llamaremos $T$, es un árbol: basta ver que no tiene
 ciclos;
  que
  es generador; y que es
  normal.
  \begin{enumerate}[1.]
  \item Sea $C$ un ciclo con vértices consecutivos $x_0, x_1, x_2 , \ldots, x_k$ en $T$. Podemos suponer que $x_k$ fue
   el último en haber sido visitado. Luego $x_0$ tuvo que haber sido el vértice que se visitó primero de $C$ y en
   consecuencia se visitó en orden: $x_0, x_1, \ldots, x_k$. Pero una vez visitado $x_k$ el algorítmo sigui corriendo
    sin volver a $x_0$ pues este ya fue visitado, pero esto significa que $C$ no puede tener la arista $x_k x_0$,
    absurdo.
  \item Supongamos que existe un vértice no visitado, luego existe un vértice sin visitar a distancia mínima de $T$,
  i.e. adyacente a $x \in T$. Luego el algorítmo tuvo que pasar por este vértice cuando volvió a $x$ por última vez.
    \item Como $T$ genera, se sigue que es normal si y solo si para todo par de vértices adyacentes en $G \setminus T$ son comparables. En efecto, supongamos que el algorítmo visitó primero a $x$ y luego a $y$, con lo cual $x \leq y$.
  \end{enumerate}
 \end{solution}

 \begin{exercise}
     Sea $\mathcal T$ un conjunto de subárboles de un árbol $T$, y $k \in \naturals$.
     \begin{enumerate}[(i)]
     \item Mostrar que si los árboles de $\mathcal T$ son disjuntos dos a dos, entonces $\bigcap_{S \in \mathcal T} S \neq \emptyset$.
     \item Mostrar que o $\mathcal T$ tiene $k$ árboles disjuntos o existe un conjunto de a lo sumo $k-1$ vérticces
     de $T$ en $\bigcap_{S \in \mathcal T} S$.
     \end{enumerate}
 \end{exercise}
 Notar que $\mathcal T$ tiene que ser finito, por ejemplo si $T$ es finito, porque si no el primer ítem fallta
 tomando una cadena (con el orden de inclusión) de conjuntos de caminos infinitos.

\begin{solution}
    \begin{enumerate}[(i)]
        \item Basta probar que la intersección de dos subárboles de $T$ es un subárbol de $T$, pues . En efecto, es ac
        íclico; veamos que también es conexo. De lo contrario, sean $x,y$ dos elementos de la intersección $S \cap S'$, de dos subárboles, en distintas componentes. Tomemos a $y$ de manera que tenga distancia mínima a $x$ en $S$, y tomemos un camino $P$ que la realice. Por otro lado, hay otro camino $yQx$ en $S'$ que debe ser internamente disjunto con $P$, de lo contrario violaríamos la minimalidad de $P$. Con lo cual, $T$ tendría un ciclo $x P y Q x$, absurdo.
        \item Por inducción en $k$. Si $k = 1$
    \end{enumerate}
\end{solution}



\begin{exercise}
    Probar que todo automorfismo de un árbol fija un vértice o una arista.
\end{exercise}
\begin{solution}
        Supongamos que el automorfismo $\varphi$ no fija ningún vértice. Veamos que fija entonces una arista. Tomemos $x \in T$ y consideremos el conjunto $\{ y \in T | x < y \}$; también consideremos $\{ x \in T | x < \varphi (x) \}$. Tomemos a $x$ maximal en este conjunto. Entonces que pasa si hacemos $\varphi (x)$? Siguen siendo comparables $\varphi (x) y \varphi^2 (x)$, más aún, $\varphi$ preserva la orientación. Con lo cual, al ser distintas,  $\varphi (x) < \varphi^2 (x)$, absurdo por maximalidad de $x$. Consecuentemente tenemos que $\varphi^2 x > \varphi (x)$. Consideremos ahora $z$ el minimo de los vértices más grandes que $x$. Veamos que $ z = \varphi (x)$ y por lo tanto $\varphi (x)$ y $x$ son adyacentes, en particular $\varphi$ preserva la arista $x \varphi(x)$. En efecto, de lo contrario tendríamos $x < z < \varphi (x)$ y como $z$ no está en el conjunto: $z > \varphi (z)$ y tenemos $\varphi (x) > z > x > \varphi (z)$
\end{solution}



\begin{exercise}
    Mostrar que en un grafo conexo los conjuntos de aristas que son minimales con la propiedad de contener una arista
     de cada árbol generador son precisamente los enlaces del grafo.
\end{exercise}
\begin{solution}
    Por un lado, un corte de $G$ tiene que tener una arista de cada árbol generador, pues el arbol es conexo y no se
    puede separar por las partes que inducen el corte; luego los enlaces son minimales con esta propiedad. Por otro
    lado, lo anterior implica que los conjuntos de aristas minimales con esta propiedad, cumplen que si son cortes
    entonces son enlaces. Luego basta probar que estos conjuntos son cortes. Para eso, usamos la Proposición \ref{},
    que dice que $F \in \mathcal B$ si y solo si $F \in \mathcal C ^\perp$, es decir, todos los cortes de $G$
    intersecan un número  par de veces a cualquier ciclo de $G$. Ahora, necesitamos el siguiente lema:
    \begin{lemma}
        Sea un $S$ un conjunto de aristas fuera de los árboles generadores $T_1,\ldots,T_r$ de $G$, entonces el conjunto
        $$
        S' = S + \sum_{f \in S \cap T'} D_f ( T') \quad \text{($D_f (T')$ es un corte fundamental)},
        $$
        con $T'$ un árbol de generador de $G$, tiene a todas sus aristas fuera de $T_1,\ldots,T_r,T'$.
    \end{lemma}
    \begin{proof}
        Antes, necesitamos observar que el conjunto $S'$ no tiene aristas en $T'$, pues de lo contrario, digamos $g \in T'$, si $g \in S$ entoncces está en un único $D_f (T)$, luego no esta en $S'$; si $g \not \in S$, luego no puede estar en ningún $D_f (T')$, pues $g \in T'$.

        Ahora, veamos que tampoco tiene aristas en $T_i$ con $1 \leq i \leq r$. En efecto, sea $g \in T = T_i \cap S'$
        con $i$ fijo, entonces sabemos por lo anterior que $g \not \in T'$, también tenemos que como $S  \cap T = \emptyset$, luego $g \not \in S$, consecuentemente  está en una
        cantidad impar de $
        D_f(T'
        )$ con $f \in S \cap T'$ porque $g \in S'$, sin embargo, tenemos por dualidad que $g \in D_f(T')$ si y solo
        si $f \in C_g (T')$, si y solo si $f \in \left (\sum_{f \in S \cap T'} D_f (T') \right ) \cap C_e (T')$ pero
        esta intersección es par porque los corrtes fundamentales están en $\mathcal B = \mathcal C^\perp$. Absurdo.
    \end{proof}

    Finalmente, consideramos aplicar el lema anterior reiteradas veces a $S := F + \sum_{f \in F \cap T} D_{f} (T)$
    con $T$ un árbol generador arbitrario: $S$ no contiene aristas en $T$. Luego se puede obtener $S' = F + F'$ un
    conjunto de
     aristas, con $F' \in \mathcal B$ pues los $D_f(T)$ son cortes fundamentales, que está fuera de todo árbol
     generador de $G$, es decir es vacío. En particular $F = F' \in \mathcal B$.
\end{solution}

















%%%%%%%%%%%%%%%%%%%%%%%%%%%%%%%%%%%%%%%%%%%%%%%%%%%%%%%%%%%%%%%%%%%

%import{nombre de carpeta/}{Nombre del archivo}
\subfile{Apendice/Apendice.tex}





%--------------------------------
\newpage

\bibliographystyle{alpha}
\bibliography{BibliografiaGRAFOS}{}
%--------------------------------







\end{document}

