\documentclass[12pt]{report}
\usepackage[utf8]{inputenc}

\usepackage[a4paper,width=150mm,top=25mm,bottom=25mm]{geometry}

\usepackage{subfiles} %esto es para modularizar el overleaf
%para usar este paquete solamente hay que usar el comando
%\subfile{}

\usepackage{graphicx}
\usepackage{framed}
\usepackage[dvipsnames]{xcolor}

\usepackage{xparse}
\usepackage{xstring}

\usepackage{stmaryrd} %para poner el comando \mapsfrom "<---|"

\usepackage{amssymb}

\usepackage{amsmath}
\usepackage{amsthm}

\usepackage{subfig}

\usepackage{mathrsfs} % para tener las fuentes \mathscr que es una letra mayuscula cursiva.

\usepackage{tikz-cd}

\usepackage{tkz-graph}%este paquete es para crear grafos con el ambiente \begin{tikzpicture}

\usepackage{graphicx}

\usepackage{caption}

\usepackage[shortlabels]{enumitem}

\usepackage[sc]{mathpazo}

\usepackage[spanish,activeacute]{babel}

\usepackage{xparse}
\usepackage{xstring}

\usepackage{braket} %para definir \set , \Set y que los conjuntos se vean mas lindos

\usepackage{mathtools}

\usepackage{mathabx}

\usepackage[shortlabels]{enumitem}

\usepackage{hyperref}
\hypersetup{
    colorlinks,
    citecolor=red,
    filecolor=red,
    linkcolor=red,
    urlcolor=red
}

%%%%%%%%%%%%%%%%%%%%%%%%%%%%%%%%%%%%%%%%%%%%%
\theoremstyle{plain}
\newtheorem{theorem}{Teorema}[section]
\newtheorem{lemma}[theorem]{Lema}
\newtheorem{proposition}[theorem]{Proposición}
\newtheorem{proposition/definition}[theorem]{Proposición/Definición}
\newtheorem{corollary}[theorem]{Corolario}
\newtheorem{conjecture}[theorem]{Conjetura}
\newtheorem{afirmacion}[theorem]{Afirmación}
\newtheorem{recuerdo}[theorem]{Recuerdo}

\theoremstyle{definition}
\newtheorem{definition}[theorem]{Definición}
\newtheorem{hypothesis}[theorem]{Hipótesis}
\newtheorem{example}[theorem]{Ejemplo}
\newtheorem{obs}[theorem]{Observación}
\newtheorem{notation}[theorem]{Notación}
\newtheorem{comentario}[theorem]{Comentario}

\newtheorem{exercise}[theorem]{Ejercicio}
%%%%%%%%%%%%%%%%%%%%%%%%%%%%%%%%%%%%%%%%%%%%%




%grupos de matrices
%SL
\newcommand{\SL}[2]{\operatorname{SL}_{#1} ( #2)}
%GL
\newcommand{\GL}[2]{\operatorname{GL}_{#1} ( #2)}

%matriz identidad
\newcommand{\Id}{\operatorname{Id}}



%enteros Z
\newcommand{\integers}{\mathbb{Z}}
%racionales
\newcommand{\rationals}{\mathbb{Q}}
%naturales
\newcommand{\naturals}{\mathbb{N}}
%reales R
\newcommand{\reals}{\mathbb{R}}
%imaginarios
\newcommand{\complex}{\mathbb{C}}
%p-adicos
\newcommand{\padics}{\mathbb{Q}_p}
%enteros p-adicos
\newcommand{\padicintegers}{\mathbb{Z}_p}

%cuerpos finitos
%Fp
\newcommand{\Fp}{\mathbb{F}_p}
%Fq
\newcommand{\Fq}{\mathbb{F}_q}



%valor absoluto p-adico
\newcommand{\abs}[1]{\left \vert #1 \right \vert}
%valor absoluto p-adico
\newcommand{\Abs}[1]{\left \vert \left \vert #1 \right \vert \right \vert}
%valuacion p-adica
\newcommand{\val}[1]{\operatorname{val} (#1)}

%Hom
\newcommand{\Hom}{\operatorname{Hom}}

%imagen y núcleo
\newcommand{\Imagen}{\operatorname{Im}}
\newcommand{\Ker}{\operatorname{Ker}}

%coker
\newcommand{\Coker}{\operatorname{Coker}}

%limite inverso
\newcommand{\liminv}{\varprojlim}


%un poco de typeset para categorias
\newcommand{\catname}[1]{{\operatorfont\textbf{#1}}}


\renewcommand{\hat}[1]{\widehat{#1}}
\renewcommand{\bar}[1]{\overline{#1}}

%declaro un comando nuevo para escribir restricción de funciones
\newcommand\rest[2]{{% we make the whole thing an ordinary symbol
  \left.\kern-\nulldelimiterspace % automatically resize the bar with \right
  #1 % the function
  \vphantom{\big|} % pretend it's a little taller at normal size
  \right|_{#2} % this is the delimiter
  }}


%%%%   COMANDO ALGEBRA CONMUTATIVA   %%%%

%altura de un ideal:
\newcommand{\height}{\textsc{height}}

%Clausura topológica
\newcommand{\closure}[1]{\overline{#1}}

%longitud de un A-modulo. Notacion: \length_A M
\newcommand{\length}{\operatorname{length}}

%Anulador de un $A$-módulo.
\newcommand{\Ann}[1]{\operatorname{Ann} (#1)}

%Cuerpo de fracciones. Notacion $\FracField A$.
\newcommand{\FracField}[1]{\operatorname{Fr} (#1)}


%%%%%%%%%%%%%%%%%%%%%%%%%%%%%%%%%%%%






%%%%   COMANDO TEORÍA DE NÚMEROS  %%%%

%Discriminante
\newcommand{\discriminant}[1]{\mathfrak{d} (#1 )}

%%%%Ideales primos%%%
%escribe una letra en notación mathfrak, para denotar a un ideal o elemento primo.

\newcommand{\primo}[1]{\mathfrak{#1}}
\newcommand{\Primo}[1]{\mathfrak{\MakeUppercase{#1}}}

%anillo de enteros O_K
\renewcommand{\O}{\mathcal{O}}
%anillo de enteros con subindice de cuerpo (input, por ejemplo $K$).
\newcommand{\integralring}[1]{O_{#1}}

%caracteristica de un cuerpo Char k
\newcommand{\Char}[1]{\operatorname{Char} #1}

%traza. Notación \trace = Tr
\newcommand{\trace}{\operatorname{Tr}}

%Traza de extensiones. Notación \Tr L K \alpha = \operatorname{Tr}_{L/K} (\alpha)
\newcommand{\Tr}[1]{\operatorname{Tr}_{L/K} (#1)} %la extension es L/K por default
\newcommand{\tr}[3]{\operatorname{Tr}_{#1/#2} (#3)}

%Norma de extensiones. Notación \Norm L K \alpha = \operatorname{N}_{L/K} (\alpha)
\newcommand{\Norm}[1]{\operatorname{N}_{L/K} (#1)}%la extension es L/K por default
\newcommand{\norm}[3]{\operatorname{N}_{#1/#2} (#3)}


%discriminante de una forma bilineal simetrica. notacion \disc{B} = \operatorname{disc} ( B)
\newcommand{\disc}[1]{\operatorname{disc} (#1)}

%%%%%%%%%%%%%%%%%%%%%%%%%%%%%%%%%%%%

%%%%%%%%%%%%%COMANDO GRAFOS%%%%%%%%%%%%%

%diámetro de un grafo
\newcommand{\diam}[1]{\operatorname{diam} (#1)}

%radio de un grafo
\newcommand{\rad}[1]{\operatorname{rad}(#1)}











%%%%%%%%%%%%%%%%%%%%%%%%%%%%%%%%%%%%



%solución
\newenvironment{solution}{\begin{proof}[Solución]}{\end{proof}}



%%%%%%%%%%%%%%%%%%%%%%%%%%%%%%%

\title{Entrega 3 - GRAFOS}
\author{Enzo Giannotta}






\begin{document}

\maketitle

\section{Teorema de Menger}
Como vimos en clase:

\begin{theorem}[Versión global de Menger]\label{Menger global}
\begin{enumerate}[(i)]
\item Un grafo es $k$-conexo si y solo si contiene $k$-caminos internamente disjuntos entre cada par de vértices.
\item Un grafo es $k$-aristaconexo si y solo si contiene $k$ caminos arista disjuntos entre cada par de vértices.
\end{enumerate}
\end{theorem}

\section{Entrega 3 - Viernes 21/04/2023}

\begin{exercise}

\begin{enumerate}

\item Para $n, m \ge 3$, determine $\kappa(G)$ cuando $G$ es: el camino en $n$ aristas $P_n$; el ciclo en $n$ aristas $C_n$; el bipartito completo $K_{m,n}$.

\item En clase enunciamos que el grafo bloque de cualquier grafo conexo es un árbol. Demuestre este resultado.

\item Sea $G$ un grafo $k$-conexo, y sea $xy$ una arista de $G$. Demuestre que  $G/xy$ es $k$-conexo si y s\'olo si $G-\{x,y\}$ es $(k-1)$-conexo.


\end{enumerate}

\end{exercise}
\begin{solution}
\begin{enumerate}
\item Utilizaré la versión global del teorema de Menger, que caracteriza el número de $k$-conexión y $k$-aristaconexión. El autor de esta entrega de ejercicios reconoce que es exagerado usar el teorema, sin embargo piensa que es una divertida y sencilla aplicación de este bello resultado.

Sea $P_n$ un camino con  $n\geq 3$ vértices, el primer ítem de \ref{Menger global} dice que $P_n$ es $1$-conexo, pues claramente entre cualquier par de vértice de un camino existe un máximo de $1$ camino. Recíprocamente, como no puede haber más de un camino internamente disjuntos entre cualquier par de vértices se sigue que $\kappa (P_n) = 1, \forall n \geq 1$.

Sea $C_n$ un cíclo con $n\geq 3$ vértices, insisto en usar \ref{Menger global}, en este caso es evidente que entre cualquier par de vértices hay dos caminos internamente disjuntos (uno por cada sentido de las agujas del reloj). Recíprocamente, no puede haber más de $3$ caminos internamente disjuntos entre cualquier par de vértice, i.e. $\kappa (G) = 2$.

Sea $K_{n,m}$ un grafo completo bipartito con $n,m \geq 3$, afirmo que $\kappa (K_{n,m}) = \min \{n,m\}$. Nuevamente aplicaremos Menger (reconociendo que se puede resolver muy fácilemnte de manera elemental).
Sin pérdida de generalidad supongamos que $n \leq m$, es decir que $n = \min \{n,m\}$, veamos entonces que $\kappa (K_{n,m}) = n$.
Escribamos $A\bigcup B = K_{n,m}$ para la partición natural de $K_{n,m}$, con $\abs A = n$ y $\abs B = m$. Por un lado, si elegimos dos vértices distintos $x$ e $y$, pueden estar en la misma parte $A$ o $B$, o estar en partes opuestas. En el primer caso, los caminos internamente disjuntos que salen de $x$ y llegan a $y$ si o si tienen que pasar por un vértice de la parte opuesta, esto dice que el número máximo $k$ de caminos internamente disjuntos entre $x,y$ es $\leq \abs A = n$ o $\leq \abs B = m$, dependiendo de si $x,y$ están en $B$ o $A$ respectivamente; es más, siempre podemos encontrar al menos $n$ caminos int. disjuntos (si $x,y \in B$) o $m$ caminos int. disjuntos (si $x,y \in A$): por ejemplo, podemos armar un camino que empiece con una arista en $x \in A$ y se conecte a cualquier vértice de $B$ y luego la segunda arista que se conecte directamente con $y \in A$; análogamente si $x,y \in B$. El otro caso es si miramos dos vértices en partes opuestas, digamos $x \in A$ e $y\in B$, aquí se ve que todo camino que sale de $y$ tiene que pasar por vértice de $A$, luego si $k$ es el número de caminos internamente disjuntos entre $x$ e $y$ se tiene que $k \leq \abs A = n$; además, efectivamente podemos construir $n$ caminos internamente disjuntos entre $x$ e $y$: empezando en $y$ construimos una arista que vaya a cualquier vértice de $A$, luego para la siguiente arista elegimos un vértide de $B$ distinto por cada camino (se puede pues $n \leq m$) y finalmente nos conectamos con $x$.
En resumen, entre cualquier par de vértices hay $n$ caminos internamente disjuntos, y hay pares de vértices en los que no pueden haber más de $n$, por ejemplo cuando $x,y$ son dos vértices en partes distintas. Con lo cual, \ref{Menger global} dice que $\kappa (K_{n,m}) = n = \min \{ n,m\}$.

\item Notación: al grafo bloque de $G$ lo denotamos por $Block(G)$. A un subgrafo conexo sin vértices de corte maximal, i.e. un bloque, lo vamos a denotar con las letras mayusculas $B,C,D$. Y denotaremos con la misma letra al vértice que inducen en el grafo $Block(G)$. A los vértices de corte los denotaremos por una letra minúscula como $x,y,z,u,v,w$ y los denotaremos de la misma manera en el grafo $Block (G)$. Quedará claro dependiendo del contexto, a qué grafo pertenece cada vértice en esta notación. Haremos el abuso de notación y llamaremos bloque tanto al subgrafo de $G$ como al vértice de $Block (G)$. Análogamente, cuando digamos vértice de corte de $Block(G)$ nos estamos refiriendo a un vértice que proviene de un vértice de corte de $G$.

Si $G$ es conexo, luego $Block(G)$ es conexo. Antes notemos que basta probar que entre dos blockes de $Block (G)$ existe un camino, pues todo vértice de corte de $Block (G)$ es adyacente a algún bloque en $Block (G)$ por definición de grafo bloque. Sean $B,B'$ dos bloques de $Block(G)$, consideremos luego a partir de un $B,B'$-camino siempre podemos construir un camino que no puede entrara y salir de un bloque más de una vez, por conexión del bloque. Este camino nos induce un camino en $Block(G)$ dado por $\tilde P : B_0 v_0 B_1 v_1 \cdots B_{r-1} v_{r-1} B_r $, donde cada bloque o vértice aparece en el orden en el cual el camino $P$ se intersecó por primera véz con estos en $G$.

Ahora vevamos que $Block(G)$ es aciclico. En efecto, supongamos que no, sea $C$ un ciclo en $Block(G)$. Como $Block(G)$ es bipartito (particionamos entre vértices de corte y bloques), no tiene ciclos impares, luego $C$ tiene al menos $4$ vértices (pueden ser cortes o bloques). Con lo cual, existen dos bloques distintos $B_1,B_2$ y dos vértices de corte distintos $v_1,v_2$ tal que podemos escribir $C : B_1 v_1 B_2 \cdots v_2 B_1$. Pero esto quiere decir que hay otro $B_1,B_2$-camino en $G$ que no pasa por $v_1$, es decir que $v_1$ no era vértice de corte, absurdo.

\item Primero notemos que $G/xy$ tiene más de $k$ vértices, si y solo si $G \setminus \{ x,y \}$ tiene más de $k-1$ vértices.

Veamos que $G/xy$ implica $G\setminus \{x,y\}$ es $(k-1)$-conexo. En efecto, sea $X$ un conjunto de menos de $k-1$ vértices en $G \setminus{x,y}$, quitarlos sigue manteniendo la conexión, pues si no quitar extos vértices y $v_{xy}$ de $G/xy$ forma una separación de $k$ vértices, que es imposible.

Recíprocamente, si tenemos un conjunto $X$ de menos de $k$ vértices en $G/xy$, entonces pueden ocurrir dos casos:
\begin{enumerate}[(i)]
\item $v_{xy} \in X$, luego debe quedar conexo, de lo contrario obtendríamos una $k-1$ separación de $G\setminus \{x,y\}$, dado por quitar los vértices de $X$ distintos de $v_{xy}$.
\item $v_{xy} \not \in X$, luego debe quedar conexo, de lo contrarío obtendríamos una $k$ separación de $G$, dado por quitar este conjunto de $k$ elementos a $G$, que es $k$-conexo.
\end{enumerate}
\end{enumerate}
\end{solution}


\end{document}

