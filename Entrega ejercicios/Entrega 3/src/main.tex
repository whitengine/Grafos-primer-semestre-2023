\documentclass[12pt]{article}
\usepackage[utf8]{inputenc}

\usepackage[a4paper,width=150mm,top=25mm,bottom=25mm]{geometry}

\usepackage{subfiles} %esto es para modularizar el overleaf
%para usar este paquete solamente hay que usar el comando
%\subfile{}

\usepackage{graphicx}
\usepackage{framed}
\usepackage[dvipsnames]{xcolor}

\usepackage{xparse}
\usepackage{xstring}

\usepackage{stmaryrd} %para poner el comando \mapsfrom "<---|"

\usepackage{amssymb}

\usepackage{amsmath}
\usepackage{amsthm}

\usepackage{subfig}

\usepackage{mathrsfs} % para tener las fuentes \mathscr que es una letra mayuscula cursiva.

\usepackage{tikz-cd}

\usepackage{tkz-graph}%este paquete es para crear grafos con el ambiente \begin{tikzpicture}

\usepackage{graphicx}

\usepackage{caption}

\usepackage[shortlabels]{enumitem}

\usepackage[sc]{mathpazo}

\usepackage[spanish,activeacute]{babel}

\usepackage{xparse}
\usepackage{xstring}

\usepackage{braket} %para definir \set , \Set y que los conjuntos se vean mas lindos

\usepackage{mathtools}

\usepackage{mathabx}

\usepackage[shortlabels]{enumitem}

\usepackage{hyperref}
\hypersetup{
    colorlinks,
    citecolor=red,
    filecolor=red,
    linkcolor=red,
    urlcolor=red
}

%%%%%%%%%%%%%%%%%%%%%%%%%%%%%%%%%%%%%%%%%%%%%
\theoremstyle{plain}
\newtheorem{theorem}{Teorema}[section]
\newtheorem{lemma}[theorem]{Lema}
\newtheorem{proposition}[theorem]{Proposición}
\newtheorem{proposition/definition}[theorem]{Proposición/Definición}
\newtheorem{corollary}[theorem]{Corolario}
\newtheorem{conjecture}[theorem]{Conjetura}
\newtheorem{afirmacion}[theorem]{Afirmación}
\newtheorem{recuerdo}[theorem]{Recuerdo}

\theoremstyle{definition}
\newtheorem{definition}[theorem]{Definición}
\newtheorem{hypothesis}[theorem]{Hipótesis}
\newtheorem{example}[theorem]{Ejemplo}
\newtheorem{obs}[theorem]{Observación}
\newtheorem{notation}[theorem]{Notación}
\newtheorem{comentario}[theorem]{Comentario}

\newtheorem{exercise}[theorem]{Ejercicio}
%%%%%%%%%%%%%%%%%%%%%%%%%%%%%%%%%%%%%%%%%%%%%




%grupos de matrices
%SL
\newcommand{\SL}[2]{\operatorname{SL}_{#1} ( #2)}
%GL
\newcommand{\GL}[2]{\operatorname{GL}_{#1} ( #2)}

%matriz identidad
\newcommand{\Id}{\operatorname{Id}}



%enteros Z
\newcommand{\integers}{\mathbb{Z}}
%racionales
\newcommand{\rationals}{\mathbb{Q}}
%naturales
\newcommand{\naturals}{\mathbb{N}}
%reales R
\newcommand{\reals}{\mathbb{R}}
%imaginarios
\newcommand{\complex}{\mathbb{C}}
%p-adicos
\newcommand{\padics}{\mathbb{Q}_p}
%enteros p-adicos
\newcommand{\padicintegers}{\mathbb{Z}_p}

%cuerpos finitos
%Fp
\newcommand{\Fp}{\mathbb{F}_p}
%Fq
\newcommand{\Fq}{\mathbb{F}_q}



%valor absoluto p-adico
\newcommand{\abs}[1]{\left \vert #1 \right \vert}
%valor absoluto p-adico
\newcommand{\Abs}[1]{\left \vert \left \vert #1 \right \vert \right \vert}
%valuacion p-adica
\newcommand{\val}[1]{\operatorname{val} (#1)}

%Hom
\newcommand{\Hom}{\operatorname{Hom}}

%imagen y núcleo
\newcommand{\Imagen}{\operatorname{Im}}
\newcommand{\Ker}{\operatorname{Ker}}

%coker
\newcommand{\Coker}{\operatorname{Coker}}

%limite inverso
\newcommand{\liminv}{\varprojlim}


%un poco de typeset para categorias
\newcommand{\catname}[1]{{\operatorfont\textbf{#1}}}


\renewcommand{\hat}[1]{\widehat{#1}}
\renewcommand{\bar}[1]{\overline{#1}}

%declaro un comando nuevo para escribir restricción de funciones
\newcommand\rest[2]{{% we make the whole thing an ordinary symbol
  \left.\kern-\nulldelimiterspace % automatically resize the bar with \right
  #1 % the function
  \vphantom{\big|} % pretend it's a little taller at normal size
  \right|_{#2} % this is the delimiter
  }}


%%%%   COMANDO ALGEBRA CONMUTATIVA   %%%%

%altura de un ideal:
\newcommand{\height}{\textsc{height}}

%Clausura topológica
\newcommand{\closure}[1]{\overline{#1}}

%longitud de un A-modulo. Notacion: \length_A M
\newcommand{\length}{\operatorname{length}}

%Anulador de un $A$-módulo.
\newcommand{\Ann}[1]{\operatorname{Ann} (#1)}

%Cuerpo de fracciones. Notacion $\FracField A$.
\newcommand{\FracField}[1]{\operatorname{Fr} (#1)}


%%%%%%%%%%%%%%%%%%%%%%%%%%%%%%%%%%%%






%%%%   COMANDO TEORÍA DE NÚMEROS  %%%%

%Discriminante
\newcommand{\discriminant}[1]{\mathfrak{d} (#1 )}

%%%%Ideales primos%%%
%escribe una letra en notación mathfrak, para denotar a un ideal o elemento primo.

\newcommand{\primo}[1]{\mathfrak{#1}}
\newcommand{\Primo}[1]{\mathfrak{\MakeUppercase{#1}}}

%anillo de enteros O_K
\renewcommand{\O}{\mathcal{O}}
%anillo de enteros con subindice de cuerpo (input, por ejemplo $K$).
\newcommand{\integralring}[1]{O_{#1}}

%caracteristica de un cuerpo Char k
\newcommand{\Char}[1]{\operatorname{Char} #1}

%traza. Notación \trace = Tr
\newcommand{\trace}{\operatorname{Tr}}

%Traza de extensiones. Notación \Tr L K \alpha = \operatorname{Tr}_{L/K} (\alpha)
\newcommand{\Tr}[1]{\operatorname{Tr}_{L/K} (#1)} %la extension es L/K por default
\newcommand{\tr}[3]{\operatorname{Tr}_{#1/#2} (#3)}

%Norma de extensiones. Notación \Norm L K \alpha = \operatorname{N}_{L/K} (\alpha)
\newcommand{\Norm}[1]{\operatorname{N}_{L/K} (#1)}%la extension es L/K por default
\newcommand{\norm}[3]{\operatorname{N}_{#1/#2} (#3)}


%discriminante de una forma bilineal simetrica. notacion \disc{B} = \operatorname{disc} ( B)
\newcommand{\disc}[1]{\operatorname{disc} (#1)}

%%%%%%%%%%%%%%%%%%%%%%%%%%%%%%%%%%%%

%%%%%%%%%%%%%COMANDO GRAFOS%%%%%%%%%%%%%

%diámetro de un grafo
\newcommand{\diam}[1]{\operatorname{diam} (#1)}

%radio de un grafo
\newcommand{\rad}[1]{\operatorname{rad}(#1)}











%%%%%%%%%%%%%%%%%%%%%%%%%%%%%%%%%%%%



%solución
\newenvironment{solution}{\begin{proof}[Solución]}{\end{proof}}



%%%%%%%%%%%%%%%%%%%%%%%%%%%%%%%

\title{Entrega 3 - GRAFOS}
\author{Enzo Giannotta}






\begin{document}

\maketitle

\section{Entrega 3 - Viernes 06/05/2023}

\begin{exercise}
\begin{enumerate}[(a)]
\item Si $M$ es un matching en un grafo $G$ tal que $\abs M < \delta (G) / 2$, entonces $M$ no es máximo. Concluya que $\alpha ' (G) \geq \delta (G)/2$ para todo grafo $G$.
\item Demuestre para todo grafo $G$ bipartito, $\alpha ' (G) \geq \delta (G)$.
\item En cada caso, encontrar una familia infinita de grafos donde se cumple la igualdad.
\end{enumerate}
\end{exercise}
\begin{solution}
\begin{enumerate}[(a)]
\item Esto equivale a probar que todo grafo $G$ tiene un matching the tamaño al menos $\lceil \delta (G)/2 \rceil$. En efecto, sabemos que todo grafo contiene un camino de longitud $\geq \delta (G)$, luego en ese camino podemos construir un matching obvio de longitud $\geq \lceil \delta (G) /2 \rceil \geq \delta (G)/2$: elegimos la primera arista (en un orden natural inducido por el camino) y luego la siguiente arista no adyacente más cercana, y repetimos el procedimiento de manera recursiva. (En un camino de longitud $n = 2k$ este algorítmo construye un matching de tamaño $k$, y en un camino de longitud $n = 2k +1$ construye uno de tamaño $k+1$). Esto implica que $\alpha ' (G) \geq \delta (G) /2$.
\item En un $X,Y$-bigrafo. Tomemos $y_0 \in Y$. Como $y$ tiene $d(y) \geq \delta (G)$ vecinos en $X$, sabemos que $X$ tiene al menos $\delta (G)$ vértices. Tomemos un subconjunto $T \subset X$ de tamaño $\delta (G)$ (por ejemplo los vecinos de $y$). Notemos que $T$ cumple la Condición de Hall. En efecto, para todo $S \subset T$ no vacío, sabemos que cualquier vértice de $S$ tiene al menos $\delta (G)$ vecinos, en particular
$$\abs {N(S)} \geq \delta (G) = \abs T \geq \abs S .$$
Con lo cual, el Teorema de Hall no dice que existe un matching de $G$ que cubre a $T$, es decir, un matching con $\abs T = \delta (G)$ aristas. Consecuentemente $\alpha ' (G) \geq \delta (G)$.
\item
        \begin{enumerate}[(a)]
        \item Consideremos la familia $G_k = (P_{2k})^{2k}$, es decir, la familia de las $2k$-potencias de caminos de longitud $2k$ con $k \in \naturals$. Recordar que la $i$-potencia de un grafo $G$ es un grafo con los mismos vértices pero donde dos vértices son adyacentes si y solo si estaban a distancia a lo más $i$ en $G$. Afirmamos que $\delta (G_k) = 2k$ y que $\alpha ' (G_k) = k$. Numerando a los vértices de $P_{2k} : x_0 x_1 \cdots x_{2k}$, tenemos que claramente $x_i$ es adyacente a $x_j$ en $G_k$, para todo $0 \leq i, j \leq 2k$, luego $G_k$ es $2k$-regular, en particular $\delta (G_k) = 2k$. Ahora, notemos que el conjunto de aristas de $G_k$ dado por $\{ e_i\}_{i = 1}^k$ con $e_i = x_{2i-1}x_{2i}$ es un matching de $G_k$ con $k$ aristas. Esto nos dice que $\alpha ' (G) \geq k$. Por otro lado, observemos que un grafo $G$ arbitrario, con un matching de $r$ aristas, se tiene que $2 r \leq \abs G$, pues el matching cubre $2 r$ vértices distintos, con lo cual $2 \alpha ' (G) \leq \abs G$. Aplicando esta observación a $G_k$, se sigue que $\alpha ' (G_k) \leq \abs {G_k}/2 = k$. En resumen, hemos visto que $\alpha ' (G_k) = k = \delta (G_k)/2$ para todo $k \in \naturals$.
        \item Consideremos ahora la familia $G_k = K_{k,k}$ de grafos bipartitos completos con dos particiones de $k$ elementos, para $k \in \naturals$. $G_k$ es $k$-regular, en particular $\delta (G_k) = k$. Por otro lado, vimos en el ítem (b) que $\alpha ' (G_k) \geq \delta (G_k) = k$. Veamos que no puede haber un matching con más de $k$ aristas. En efecto, todo matching de $G_k$ cubre a lo más $k$ vértices en una de las particiones de $K_{k,k} = G_k$. Esto concluye la demostración de que $\alpha ' (G_k) = k = \delta (G_k)$, para todo $k \in \naturals$.
        \end{enumerate}
\end{enumerate}
\end{solution}


\begin{exercise}
Demuestre que un grafo $G$ es bipartito si y solo si $\alpha ' (H) = \beta (H)$ para todo subgrafo $H$ de $G$.
\end{exercise}

\begin{solution}
En clase vimos la demostración del Teorema de König-Egevary, que dice que vale la igualdad $\alpha ' (G) = \beta (G)$ para todo grafo bipartito $G$, luego como todo subgrafo $H$ de $G$ es también bipartito, salvo por el subgrafo trivial donde no hay nada que probar, se tiene una de las implicaciones del ejercicio.

Veamos la recíproca, es decir, veamos que si un grafo $G$ cumple la igualdad para todo subgrafo $H$ de $G$, entonces es bipartito. En efecto, supongamos por el absurdo que $G$ no es bipartito, i.e. $G$ contiene un ciclo impar $H$. Llegaremos a un absurdo si logramos probar que para todo ciclo impar, no se cumple la igualdad del enunciado.

Así es, pues si escribimos $C: x_0 \cdot x_1 \cdot x_2 \cdots x_{2k} \cdot x_0$ para cualquier ciclo impar con $k \geq 1$, entonces $\beta (C) = k+1$ ya que los vértices de índice par cubren todas las aristas y con $k$ vértices no se pueden cubrir todas las aristas, pues cada vértice cubre $2$ aristas de $C$ pero $C$ tiene $2k+1$ aristas; por otro lado, sabemos que en todo grafo $\alpha ' (C) \leq \beta (C)$, entonces basta ver que no hay un mathcing en $C$ de tamaño $k+1$: como notamos en el ejercicio anterior, en un matching con $r = k+1$ aristas, $2 r \leq \abs C = 2k +1$, imposible. Esto prueba que $G$ no puede tener ciclos impares.
\end{solution}


\begin{exercise}
Sea $G$ un $X,Y$-bigrafo tal que $\abs X = \alpha (G)$, donde $\alpha (G)$ es el tamaño máximo de un conjunto de vértices independiente de $G$. Demuestre que $G$ tiene un matching que cubre a $Y$.
\end{exercise}
\begin{solution}
Notemos primero que la igualdad $\abs X = \alpha (G)$ nos dice que $\abs Y \leq \abs X$, ya que $Y$ también es un conjunto independiente. Para probar que $G$ tiene un matching que cubre a $Y$, esto equivale a probar que todos los subconjuntos de $Y$ cumplen la Condición de Hall. En efecto, supongamos por el absurdo que no, es decir, existe un subconjunto $S \subset Y$ no vacío de tamaño mínimo tal que no cumple la condición de Hall. Claramente $\abs S > 1$, pues si no $S = \{y\}$ con  $y$ sin vecinos en $X$, pero luego $X \cup S$ es independiente y $\alpha (G) > \abs X$. Ahora tomemos $y \in S$, y notemos $S' = S \setminus \{y \} \neq \emptyset$. Por minimalidad de $S$, el conjunto $S'$ cumple la condición de Hall, i.e. existe un matching de $G$ que cubre a $S'$; más aún, $\abs {N(S')} = \abs {S'} $, de lo contrario $S$ cumpliría la condición de Hall. Consideramos ahora el conjunto $X \setminus N(S') \cup S = X \setminus N(S') \cup S' \cup \{y\}$ de $G$ de tamaño $\abs X +1$. Por hipótesis, no puede ser independiente, y por construcción, debe ser que $y$ tiene un vecino en $X \setminus N(S')$, es decir, $\abs {N (S)} \geq \abs{N(S')} + 1 = \abs {S'} + 1 = \abs S$, lo cual es imposible porque $S$ no cumplía la condición de Hall.
\end{solution}


\begin{exercise}
Calcular $\alpha ' (K_n)$ para todo $n \geq 3$. (Notar que $\alpha ' (K_2) = 1$).
\end{exercise}
\begin{solution}
Vamos a probar por inducción que
\begin{equation}\label{eq:ejercicio random 1}
    \alpha ' (K_{n+1}) = \alpha ' (k_{n}) + 1, \quad \forall n \geq 3.
\end{equation}
Esto, junto con el hecho de que $\alpha ' (K_3) = 1$, prueba la afirmación:
\[
    \alpha ' (K_n) = n-2, \quad \forall n \geq 3.
\]

Claramente vale \eqref{eq:ejercicio random 1} para $n = 3$. En general, tomando un vértice $x \in K_{n+1}$ y considerando $K_n = K_{n+1} \setminus \{x \}$ se ve por inducción que agregando una arista incidente a $x$ a un matching de $K_n$ de tamaño $\alpha ' (K_n)$, obtenemos un matching de $K_{n+1}$ una unidad mayor, i.e. $\alpha ' (K_n) + 1 \leq \alpha ' (K_{n+1})$. Por otro lado, un matching de $K_{n+1}$ de tamaño $\alpha ' (K_{n+1})$ se convierte en un matching de $K_n$ si quitamos un vértice de $K_{n+1}$ cubierto por él, i.e. $\alpha ' (K_{n+1}) - 1 \leq \alpha ' (K_n)$. Probando así la igualdad \eqref{eq:ejercicio random 1}.
\end{solution}

\end{document}

