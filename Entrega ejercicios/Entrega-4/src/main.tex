\documentclass[12pt]{article}


\usepackage{fouriernc}%la fuente
%\usepackage[sc]{mathpazo} %antigua fuente

\usepackage[utf8]{inputenc}

\usepackage[a4paper,width=150mm,top=25mm,bottom=25mm]{geometry}



\usepackage{subfiles} %esto es para modularizar el overleaf
%para usar este paquete solamente hay que usar el comando
%\subfile{}

\usepackage{graphicx}
\usepackage{framed}
\usepackage[dvipsnames]{xcolor} %agrega mas colores para xcolor.

\usepackage{xparse}
\usepackage{xstring}

\usepackage{stmaryrd} %para poner el comando \mapsfrom "<---|"

\usepackage{amssymb}

\usepackage{amsmath}

\usepackage{subfig}

\usepackage{mathrsfs} % para tener mas tipos de texto: \mathscr que es una letra mayuscula cursiva.

\usepackage{tikz-cd}

\usepackage{tkz-graph}%este paquete es para crear grafos con el ambiente \begin{tikzpicture}

\usepackage{caption}

\usepackage[shortlabels]{enumitem}

\usepackage{mathabx}
\let\widering\relax %esto es porque hay problemas con el comando \widering que se define en la fuenta fouriernc y en el paquete \usepackage{mathabx}

\usepackage[spanish,activeacute]{babel}

\usepackage{xparse}
\usepackage{xstring}

\usepackage{braket} %para definir \set , \Set y que los conjuntos se vean mas lindos

\usepackage{mathtools}

\usepackage[shortlabels]{enumitem}

\usepackage{hyperref}
\hypersetup{
    colorlinks,
    citecolor=red,
    filecolor=red,
    linkcolor=red,
    urlcolor=red
}

%%%%%%%%%%%%%%%%%%%%%%%%%%%%%%%%%%%%%%%%%%%%%
\usepackage{amsthm}

\theoremstyle{plain}
\newtheorem{theorem}{Teorema}[section]
\newtheorem{lemma}[theorem]{Lema}
\newtheorem{proposition}[theorem]{Proposición}
\newtheorem{proposition/definition}[theorem]{Proposición/Definición}
\newtheorem{corollary}[theorem]{Corolario}
\newtheorem{conjecture}[theorem]{Conjetura}
\newtheorem{afirmacion}[theorem]{Afirmación}
\newtheorem{recuerdo}[theorem]{Recuerdo}

\theoremstyle{definition}
\newtheorem{definition}[theorem]{Definición}
\newtheorem{hypothesis}[theorem]{Hipótesis}
\newtheorem{example}[theorem]{Ejemplo}
\newtheorem{obs}[theorem]{Observación}
\newtheorem{notation}[theorem]{Notación}
\newtheorem{remark}[theorem]{Comentario}


%por alguna razon el teorema $warning  est aen uso, asi que lo remuevo de maqnera trucha
\newtheorem{warn}[theorem]{\textbf{ADVERTENCIA}}
\renewenvironment{warning}{\begin{warn}}{\end{warn}}

%crear ejercicio
\newtheorem{exercise}[theorem]{Ejercicio}
%solución
\newenvironment{solution}{\begin{proof}[Solución]}{\end{proof}}





%como crear un nuevo ambiente de teorema o proposición que este sobreado con un recuadro de "color". primero hacemos

%\newenvironment{Theorem}{\colorlet{shadecolor}{color} \begin{shaded} \begin{theorem} }{ \end{theorem} \end{shaded} }

%Notar que primero hay que definir el color del sobreado con el comando
%"\colorlet{shadecolor}{color}" y luego hay que usar el environment "shaded". Adentro de este ponemos el environment que queremos, en nuestro caso queremos "pintar" el environment "\begin{theorem}".


%se puede cambiar la tonalidad de un color "yellow!80" es el color amarillo pero al 80%  y el 20% es mezclado con blanco, i.e. está aclarado. Pero "yellow!80!Black" es 80% amarillo y 20% negro, i.e. es obscurecido 20%.

\newenvironment{Definition}{\colorlet{shadecolor}{Apricot!12} \begin{shaded} \begin{definition} }{ \end{definition} \end{shaded} }

\newenvironment{Example}{\colorlet{shadecolor}{Goldenrod!16} \begin{shaded} \begin{example}}{ \end{example} \end{shaded}}

\newenvironment{Remark}{\colorlet{shadecolor}{Orchid!12} \begin{shaded} \begin{remark}}{ \end{remark} \end{shaded}}

\newenvironment{Warning}{\colorlet{shadecolor}{red!12} \begin{shaded} \begin{warning}}{ \end{warning} \end{shaded}}

\newenvironment{Conjecture}{\colorlet{shadecolor}{magenta!16} \begin{shaded} \begin{conjecture}}{ \end{conjecture} \end{shaded}}

\newenvironment{Theorem}{\colorlet{shadecolor}{OliveGreen!18} \begin{shaded} \begin{theorem}}{ \end{theorem} \end{shaded}}

\newenvironment{Lemma}{\colorlet{shadecolor}{LimeGreen!12} \begin{shaded} \begin{lemma}}{ \end{lemma} \end{shaded}}

\newenvironment{Proposition}{\colorlet{shadecolor}{Green!12} \begin{shaded} \begin{proposition}}{ \end{proposition}\end{shaded}}

\newenvironment{Corollary}{\colorlet{shadecolor}{TealBlue!16} \begin{shaded} \begin{corollary}}{ \end{corollary} \end{shaded}}

\newenvironment{Obs}{\colorlet{shadecolor}{Dandelion!22} \begin{shaded} \begin{obs}}{ \end{obs} \end{shaded}}

\newenvironment{Exercise}{\colorlet{shadecolor}{Lavender!12} \begin{shaded} \begin{exercise}}{ \end{exercise} \end{shaded}}

%%%%%COLORES%%%%%%%%%%%%
%Hay varios comandos del paquete Xcolor:
%\color{blue,green,red,yellow,orange,black,white,pink,purble,etc...} hace que todo el bloque de texto se transforme en este color, se puede encerrar entre {} el bloque de texto que uno quiere colorear
%\textcolor{color}{text} escribe el texto "text" en "color".
%\colorbox{color}{text} pinta un rectangulo de "color" detrás del "text".
%\shaded



%lista de colores base de xcolor, como son colores de la extension del paquetem, empiezan con la primera letra mayuscula: si usaramos solo el paquete {xcolor} entonces no sería necesario.

%red, Green (fluorecente), Blue (muy obscuro), Cyan, Magenta, Yellow, Black, Gray, lightgray, White, darkgray, lightgray, Brown, lime (este verde mas lindo manzana), olive (marron verdoso feo), Orange, pink, Purple, teal (verde marino), Violet

%marco los colores lindos: red, Cyan, Magenta, Yellow, Black, Gray, White,  lime, Orange, pink, teal, Violet

%Colores que incluye el paquete dvipsnames: Apricot (color beige), Brown, Goldenrod, JungleGreen, Salmon, Lavender, SpringGreen, Turquoise, Plum, Emerald, BurntOrange (naranja piola), ForestGreen (verde oscuro), BrickRed (rojo obscuro)


\newcommand{\red}[1]{\textcolor{BrickRed}{#1}}

			\newcommand{\comentario}[1]{\red{#1}}

\newcommand{\green}[1]{\textcolor{SpringGreen}{#1}}

\newcommand{\blue}[1]{\textcolor{Cyan}{#1}}

\newcommand{\darkblue}[1]{\textcolor{Cyan!70!Black}{#1}}

\newcommand{\yellow}[1]{\textcolor{yellow!80!Black}{#1}} %se puede cambiar la tonalidad de un color "yellow!80" es el color amarillo pero al 80%  y el 20% es mezclado con blanco, i.e. está aclarado. Pero "yellow!80!Black" es 80% amarillo y 20% negro, i.e. es obscurecido 20%.

\newcommand{\black}[1]{\textcolor{Black}{#1}}

\newcommand{\gray}[1]{\textcolor{Gray}{#1}}

\newcommand{\purple}[1]{\textcolor{Purple}{#1}}

\newcommand{\beige}[1]{\textcolor{Apricot}{#1}}

\newcommand{\darkgreen}[1]{\textcolor{ForestGreen}{#1}}

\newcommand{\pink}[1]{\textcolor{Lavender}{#1}}

\newcommand{\salmon}[1]{\textcolor{Salmon}{#1}}

\newcommand{\brown}[1]{\textcolor{RawSienna!50!Black}{#1}}

\newcommand{\white}[1]{\textcolor{White}{#1}}

\newcommand{\orange}[1]{\textcolor{BurntOrange}{#1}}













%%%%%%%%%%%%%%%%%%%%%%%%%%%%%%%%%%%%%%%%%%%%%




%grupos de matrices
%SL
\newcommand{\SL}[2]{\operatorname{SL}_{#1} ( #2)}
%GL
\newcommand{\GL}[2]{\operatorname{GL}_{#1} ( #2)}

%matriz identidad
\newcommand{\Id}{\operatorname{Id}}



%enteros Z
\newcommand{\integers}{\mathbb{Z}}
%racionales
\newcommand{\rationals}{\mathbb{Q}}
%naturales
\newcommand{\naturals}{\mathbb{N}}
%reales R
\newcommand{\reals}{\mathbb{R}}
%imaginarios
\newcommand{\complex}{\mathbb{C}}
%p-adicos
\newcommand{\padics}{\mathbb{Q}_p}
%enteros p-adicos
\newcommand{\padicintegers}{\mathbb{Z}_p}

%cuerpos finitos
%Fp
\newcommand{\Fp}{\mathbb{F}_p}
%Fq
\newcommand{\Fq}{\mathbb{F}_q}



%valor absoluto p-adico
\newcommand{\abs}[1]{\left \vert #1 \right \vert}
%valor absoluto p-adico
\newcommand{\Abs}[1]{\left \vert \left \vert #1 \right \vert \right \vert}
%valuacion p-adica
\newcommand{\val}[1]{\operatorname{val} (#1)}

%Hom
\newcommand{\Hom}{\operatorname{Hom}}

%imagen y núcleo
\newcommand{\Imagen}{\operatorname{Im}}
\newcommand{\Ker}{\operatorname{Ker}}

%coker
\newcommand{\Coker}{\operatorname{Coker}}

%limite inverso
\newcommand{\liminv}{\varprojlim}


%un poco de typeset para categorias
\newcommand{\catname}[1]{{\operatorfont\textbf{#1}}}


\renewcommand{\hat}[1]{\widehat{#1}}
\renewcommand{\bar}[1]{\overline{#1}}

%declaro un comando nuevo para escribir restricción de funciones
\newcommand\rest[2]{{% we make the whole thing an ordinary symbol
  \left.\kern-\nulldelimiterspace % automatically resize the bar with \right
  #1 % the function
  \vphantom{\big|} % pretend it's a little taller at normal size
  \right|_{#2} % this is the delimiter
  }}


%%%%   COMANDO ALGEBRA CONMUTATIVA   %%%%

%altura de un ideal:
\newcommand{\height}{\textsc{height}}

%Clausura topológica
\newcommand{\closure}[1]{\overline{#1}}

%longitud de un A-modulo. Notacion: \length_A M
\newcommand{\length}{\operatorname{length}}

%Anulador de un $A$-módulo.
\newcommand{\Ann}[1]{\operatorname{Ann} (#1)}

%Cuerpo de fracciones. Notacion $\FracField A$.
\newcommand{\FracField}[1]{\operatorname{Fr} (#1)}


%%%%%%%%%%%%%%%%%%%%%%%%%%%%%%%%%%%%






%%%%   COMANDO TEORÍA DE NÚMEROS  %%%%

%Discriminante
\newcommand{\discriminant}[1]{\mathfrak{d} (#1 )}

%%%%Ideales primos%%%
%escribe una letra en notación mathfrak, para denotar a un ideal o elemento primo.

\newcommand{\primo}[1]{\mathfrak{#1}}
\newcommand{\Primo}[1]{\mathfrak{\MakeUppercase{#1}}}

%anillo de enteros O_K
\renewcommand{\O}{\mathcal{O}}
%anillo de enteros con subindice de cuerpo (input, por ejemplo $K$).
\newcommand{\integralring}[1]{O_{#1}}

%caracteristica de un cuerpo Char k
\newcommand{\Char}[1]{\operatorname{Char} #1}

%traza. Notación \trace = Tr
\newcommand{\trace}{\operatorname{Tr}}

%Traza de extensiones. Notación \Tr L K \alpha = \operatorname{Tr}_{L/K} (\alpha)
\newcommand{\Tr}[1]{\operatorname{Tr}_{L/K} (#1)} %la extension es L/K por default
\newcommand{\tr}[3]{\operatorname{Tr}_{#1/#2} (#3)}

%Norma de extensiones. Notación \Norm L K \alpha = \operatorname{N}_{L/K} (\alpha)
\newcommand{\Norm}[1]{\operatorname{N}_{L/K} (#1)}%la extension es L/K por default
\newcommand{\norm}[3]{\operatorname{N}_{#1/#2} (#3)}


%discriminante de una forma bilineal simetrica. notacion \disc{B} = \operatorname{disc} ( B)
\newcommand{\disc}[1]{\operatorname{disc} (#1)}

%%%%%%%%%%%%%%%%%%%%%%%%%%%%%%%%%%%%




%%%%%%%%%%%%%COMANDO GRAFOS%%%%%%%%%%%%%

%\ceil funcion techo
\newcommand{\ceil}[1]{\left\lceil #1  \right\rceil}

%\floor funcion piso
\newcommand{\floor}[1]{\left\lfloor #1  \right\rfloor}

%diámetro de un grafo
\newcommand{\diam}[1]{\operatorname{diam} (#1)}

%radio de un grafo
\newcommand{\rad}[1]{\operatorname{rad}(#1)}

%Kappa:
\newcommand{\Kappa}{\mathcal{K}}

%Defecto:
\newcommand{\defecto}[1]{\mathrm{df}(#1)}











\title{Entrega 4 - GRAFOS}
\author{Enzo Giannotta}






\begin{document}

\maketitle

\section{Entrega 4 - Viernes 12/05/2023}

\begin{obs}\label{obs}
En general, sea $G$ un grafo con un matching máximo $M$, y sea $U$ el conjunto de vértices de $G$ que no están cubiertos por $M$. Luego si $u \in U$, implica que todos los vecinos de $u$ están cubiertos por $M$.
\end{obs}
\begin{proof}
En efecto, pues de lo contrario, si $u ' \in U$ es un vecino de $u$, agregando la arista $uu'$ a $M$ obtendríamos un matching más grande.
\end{proof}

\begin{exercise}
Sea $G$ un grafo con grado máximo $\Delta (G) = k$. Sea $M$ un matching máximo de $G$. Para $k \geq 3$, demuestre que el número de aristas que une vértices cubiertos por $M$ a vértices no cubiertos por $M$ es a lo más $(k-1) \abs M$.
\end{exercise}

\begin{solution}
Sea $S$ el conjunto de aristas que unen vértices cubiertos por $M$ a vértices no cubiertos por $M$. Llamemos $U$ al conjunto de vértices que no están cubiertos por $M$. Tenemos que para todo $u \in U$ los vecinos de $u$ son todos vértices cubiertos por $M$ por la Observación \ref{obs}.

\begin{definition}
Sea $u \in U$ con $U$ como recién. Sea $e$ una arista de $M$ que tiene uno de sus extremos vecino a $u$. Si $e$ tiene alguno de sus extremos adyacente a $u$, diremos que $e$ es \textbf{vecina} de $u$ y diremos que $u$ es \textbf{vecino} de $e$. Diremos que $u$ es \textbf{compañero} de $e$ si es adyacente a ambos extremos de $e$. De lo contrario, diremos que es \textbf{no compañero}, y notaremos como $x_u$ al único vértice de $e$ adyacente a $u$.
\end{definition}

\begin{lemma}
Para $e \in M$ fijo pueden ocurrir dos casos disjuntos:
\begin{enumerate}
\item $e$ tiene solamente un único vecino en $U$, el cual es compañero de $e$.
\item Todos los vecinos $u \in U$ de $e$ son \textit{no compañeros} ($e$ podría no tener vecinos en $U$) adyacentes a un único extremo de $e$.
\end{enumerate}
\end{lemma}
\begin{proof}
Supongamos que $e \in M$ tiene algún vecino $u \in U$. Si $u$ es el único vecino, o estamos en el primer caso o estamos en el segundo. Si $u$ no es único, es decir, existe otro vértice $v \in U$ vecino de $e$, entonces si o si $u$ y $v$ deben ser ambos adyacentes al mismo vértice $x$ o $y$, donde $e = xy$. En efecto, de lo contrario si $u$ es adyacente a $x$ y $v$ es adyacente a $y$, entonces como $u,v$ no son extremos de ninguna arista de $M$, reemplazando la arista $e \in M$ por las aristas $xu$ e $yv$, obtenemos un matching más grande que $M$, absurdo. Esta misma demostración prueba que $u$ y $v$ no pueden tener más de un vecino que sea extremo de $e$, es decir, ambos son no compañeros. Con lo cual, si aplicamos este razonamiento a todos los vecinos de $e$ en $U$, deben ser todos no compañeros y adyacentes a un único extremo de $e$, i.e. estamos en el segundo caso.

Es claro que estos dos casos son disjuntos.
\end{proof}

El lema anterior nos permite contar de la siguiente manera: para cada $e \in M$ nos fijamos si estamos en el caso 1. o en el caso 2. En el primer caso contamos solamente un vecino de $e$ en $U$, i.e. $2 \leq k-1$ aristas de $S$. En el segundo caso, todos los vecinos de $e =xy$ en $U$ son adyacentes a un único extremo $x$ o $y$, digamos $x$, luego $e$ puede tener a lo más $k-1$ vecinos en $U$ (no contamos a $y$), i.e. contamos $\leq k-1$ aristas de $S$. Juntando ambos casos disjuntos, nos queda que $\abs S \leq (k-1) \abs M$.
\end{solution}




\begin{exercise}
Dos personas juegan un juego en un grafo: se alternan para elegir vértices $v_1,v_2, \ldots$ de modo que para todo $i \geq 2$, el vértice $v_i$ es adyacente al veŕtice $v_{i-1}$ y no ha sido escogido antes. El último jugador capaz de escoger un vértice gana.
\begin{enumerate}[(a)]
\item Demuestre que el segundo jugador tiene una estrategia ganadora si el grafo tiene un matching perfecto.
\item Demuestre que el primer jugador tiene una estrategia ganadora si el grafo no posee un matching perfecto.
\end{enumerate}
\end{exercise}
\begin{solution}
\begin{enumerate}[(a)]
\item Sea $M$ un matching perfecto de $G$. Afirmo que la estrategia ganadora del segundo jugador es elegir el extremo opuesto de la arista de $M$ cuyo extremo ha sido escogido por el primer jugador en el anterior turno. En efecto, supongamos que esta estrategia falló. Es decir, ocurrió una cantidad impar elecciones de vértices de $G$: $v_1, v_2, v_3, v_4, \ldots, v_{2k -1}$ para algún $k \in \naturals$ y $v_{2k-1}$ no tiene vecinos sin visitar. (Notar que $k \geq 2$, pues el grafo está cubierto por un conjunto de aristas no vacío $M$). Ahora, la estrategia del segundo jugador implica que las aristas $e_1 := v_1 v_2, e_2 := v_3 v_4 , \ldots, e_{k-1} := v_{2k-3} v_{2k-2}$ pertenecen al matching $M$. Luego $v_{2k-1}$ pertenece a una arista de $M$ distinta de $e_1, \ldots, e_{k-1}$, es decir, $v_{2k-1}$ tiene un vecino distinto sin escoger: el extremo opuesto de una arista de $M$ que cubre a $v_{2k-1}$, absurdo. Con lo cual, esta era una estrategia ganadora para el segundo jugador.

 \item Sea $M$ un matching máximo de $G$, con $G$ sin matching perfecto. Sea $U$ el conjunto de vértices de $G$ que no están cubiertos por $M$. Notemos por $U_k$ al conjunto de vértices que no están cubiertos por las aristas de $M_k := M \cap E(G_k)$, donde $G_k$ es el grafo obtenido a partir de $G$ luego quitar todos los vértices escogidos por ambos jugadores previos al $k$-ésimo turno del primer jugador.
  La estrategia del primer jugador será siempre elegir un vértice de $U_k$ en su $k$-ésimo turno. A priori no sabemos que siempre se pueda escoger un vértice de $U_k$. Sin embargo, veremos que efectivamente se puede. Más precisamente, probaremos por inducción la afirmación más fuerte:

\begin{proposition}\label{proposicion}
Para todo $k \geq 1$, se tiene que si el primer jugador no ganó en su $(k-1)$-ésimo turno, entonces
\begin{enumerate}[(i)]
\item $M_k$ es un matching máximo de $G_k$;
\item Todos los vecinos de cualquier elemento $u$ de $U_k$ están cubiertos por una arista de $M_k$.
\item El primer jugador puede escoger un vértice de $U_k$ en su $k$-ésimo turno.
\end{enumerate}
\end{proposition}

Necesitamos un lema previo:

\begin{lemma}\label{lema}
 En general, si $M$ es un matching máximo de un grafo $G$, y $x$ es un extremo de una arista $e$ de $M$, e $y$ es un vértice adyacente a $x$ que no está cubierto por $M$. Entonces el grafo $G' = G \setminus \{x,y\}$ tiene a $M' = M \cap E(G')$ como matching máximo.
\end{lemma}
\begin{proof}
 Supongamos que no, es decir que existe un matching $W$ de $G'$ con más aristas que $M'$, es decir $\abs {M'} \leq \abs W - 1$. Notar que $\abs {M'} = \abs M - 1$ porque borramos dos véŕtices $x,y$: donde $x$ solamente es extremo de una arista de $M$ por ser un matching, e $y$ no era extremo de ninguna arista de $M$ por cómo lo elegimos. Esto implica que $\alpha ' (G) - 1 \leq \abs W - 1$. Luego como $W$ también es un matching de $G$, debe ser que $\abs W \leq \alpha ' (G)$, con lo cual $\alpha ' (G) = \abs W$ y $W$ es un matching máximo de $G$. Sean $x$ y $e \in M$ como al principio. Por construcción de $W$, tenemos que $e = xy$ no está en $W$, más aún, esta arista es independiente de $W$, luego $W \cup \{e \}$ es un matching de $G$ de tamaño $\alpha ' (G) + 1$, absurdo. Esto prueba que $M'$ es un matching máximo de $G'$.
\end{proof}

Estamos ahora en condiciones de probar la Proposición \ref{proposicion}:
\begin{proof}
Si $k = 1$, como $M_k = M$, $G_k = G$ y $U_k = U$, no hay nada que probar en (i); (ii) se sigue inmediatamente de la Observación \ref{obs}; (iii) se sigue de que $G_k$ no tiene matching perfecto. En general, si el primer jugador no ganó en su $k$-ésimo turno, $M_{k+1}$ es igual a $M_k \cap E(G_{k+1})$, donde notemos que $G_{k+1} = G_k \setminus \{x_k, y_k\}$, y $x_k$ es el vértice escogido por el primer jugador en el $k$-ésimo turno $y_k$ el vértice (adyacente) escogido subsecuentemente por el segundo jugador, más aún, por hipótesis inductiva $x_k \in U_k$ es un vértice no cubierto por el matching máximo $M_k$ de $G_k$, con lo cual $y_k$ es adyacente a $x_k$ y está cubierto por una arista de $M_k$ y luego por el Lema \ref{lema} $M_{k+1}$ es un matching máximo de $G_{k+1}$. Esto prueba (i). Por la Observación \ref{obs}, los vecinos de todo $u \in U_{k+1}$ tienen que estar cubiertos por el matching máximo $M_{k+1}$, probando así (ii). Como $y_k$ esta cubierto por una arista de $M_k$, el otro extremo de esta arista puede ser escogido por el primer jugador, y además debe estar en $U_{k+1}$, pues la única arista de $M$ que incide en este extremo fue eliminado de $G_k$ (recordar que $M_{k+1} = M_k \cap E(G_k) = M \cap E(G_{k+1})$). Así, se sigue (iii).
\end{proof}

Finalmente, el último ítem de la Proposición \ref{proposicion} dice que si el primer jugador no ganó en su $k$-ésimo turno, luego puede escoger un vértice en su $(k+1)$-ésimo turno. Esto significa que el primer jugador siempre va a ser el último en escoger un vértice, es decir, cuando el juego eventualmente termine, el primer jugador ganará: la estrategia es ganadora.
\end{enumerate}
\end{solution}














\end{document}

